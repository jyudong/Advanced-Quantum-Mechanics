\documentclass[reqno,a4paper,12pt]{amsart}

\usepackage{amsmath,amssymb,amsthm,geometry,xcolor,soul,graphicx}
\usepackage{titlesec}
\usepackage{enumerate}
%\usepackage{lipsum} used a paragraph to test the environment 
\usepackage{listings}
%\RequirePackage[most]{tcolorbox}
\usepackage{braket}
\allowdisplaybreaks[4] %align公式跨页
\usepackage{xeCJK}
\setCJKmainfont[AutoFakeBold = true]{Kai}
\geometry{left=0.7in, right=0.7in, top=1in, bottom=1in}

\renewcommand{\baselinestretch}{1.3}

\title{高等量子力学第十二次作业}
\author{董建宇 ~~ 202328000807038}

%\setlength{\parindent}{2pt}

\begin{document}

\maketitle

\begin{enumerate}[1.]

\item Prove that the d'Alembert operator $\square = \frac{1}{c^2}\frac{\partial^2}{\partial t^2} - \nabla^2$ is invariant under the Lorentz transform.

\begin{proof}
d'Alembert可以写作:
\[
	\square = \frac{1}{c^2}\frac{\partial^2}{\partial t^2} - \nabla^2 = \partial_\mu \partial^\mu = g_{\mu\nu} \partial^\mu \partial^\nu.
\]

经过洛伦兹变换后,有:
\[
	\square' = L_\lambda^\mu g_{\mu\nu} L_\rho^\nu L_\mu^\lambda \partial^\mu L_\nu^\rho \partial^\nu = g_{\lambda\rho} \partial^\lambda \partial^\rho = \partial_\lambda \partial^\lambda = \square.
\]

即d'Alembert算符保持Lorentz不变。

\end{proof}

\medskip

\item Starting from the Dirac equation, verify the equation is consistent with KG equation. We can get KG by multiplying the Dirac equation 
\[
	\frac{1}{c} \frac{\partial \psi}{\partial t} + \sum_{k=1}^3 \alpha^k \frac{\partial \psi}{\partial x_k} + i\frac{mc}{\hbar}\beta\psi = 0.
\]

by the operator
\[
	\frac{1}{c}\frac{\partial }{\partial t} - \sum_{l=1}^3 \alpha^l \frac{\partial }{\partial x_l} - i\frac{mc}{\hbar}\beta.
\]

\begin{proof}
可以计算:
\begin{align*}
	&\left( \frac{1}{c}\frac{\partial }{\partial t} - \sum_{l=1}^3 \alpha^l \frac{\partial }{\partial x_l} - i\frac{mc}{\hbar}\beta \right) \left( \frac{1}{c}\frac{\partial }{\partial t} + \sum_{k=1}^3 \alpha^k \frac{\partial }{\partial x_k} + i\frac{mc}{\hbar}\beta \right) \\
	=& \frac{1}{c^2} \frac{\partial^2}{\partial t^2} - \left( \sum_{k=1}^3 \alpha^k \frac{\partial }{\partial x_k} + i\frac{mc}{\hbar}\beta \right)^2 \\
	&+ \frac{1}{c}\frac{\partial}{\partial t} \left( \sum_{k=1}^3 \alpha^k \frac{\partial }{\partial x_k} + i\frac{mc}{\hbar}\beta \right) - \left( \sum_{l=1}^3 \alpha^l \frac{\partial }{\partial x_l} + i\frac{mc}{\hbar}\beta \right) \frac{1}{c} \frac{\partial}{\partial t}.
\end{align*}

其中,由于
\[
	\left[\frac{\partial}{\partial t}, \sum_{k=1}^3 \alpha^k \frac{\partial }{\partial x_k} + i\frac{mc}{\hbar}\beta \right] = 0.
\]

后两项和等于0。

利用
\[
	\{\alpha_i, \alpha_j\} = 2\delta_{ij}, \ \ \{\alpha_i, \beta\} = 0, \ \ \beta^2 = \mathbf{I}.
\]

可以进一步计算得到:
\begin{align*}
	&\left( \frac{1}{c}\frac{\partial }{\partial t} - \sum_{l=1}^3 \alpha^l \frac{\partial }{\partial x_l} - i\frac{mc}{\hbar}\beta \right) \left( \frac{1}{c}\frac{\partial }{\partial t} + \sum_{k=1}^3 \alpha^k \frac{\partial }{\partial x_k} + i\frac{mc}{\hbar}\beta \right) \\
	=& \frac{1}{c^2} \frac{\partial^2}{\partial t^2} - \left( \sum_{k=1}^3 \alpha^k \frac{\partial }{\partial x_k} + i\frac{mc}{\hbar}\beta \right)^2 \\
	=& \frac{1}{c^2} \frac{\partial^2}{\partial t^2} - \nabla^2 + \frac{m^2c^2}{\hbar^2}.
\end{align*}

即有:
\[
	\left( \frac{1}{c^2} \frac{\partial^2}{\partial t^2} - \nabla^2 + \frac{m^2c^2}{\hbar^2} \right) \psi = 0.
\]

即得到了Klein-Gordon方程。

\end{proof}

\medskip

\item Prove $[H, \boldsymbol{\varSigma} \cdot \boldsymbol{p}] = 0$.

\begin{proof}
哈密顿量为:
\[
	H = -i\hbar c \boldsymbol{\alpha \cdot \nabla} + mc^2\beta.
\]

则可以计算:
\[
	[H, \boldsymbol{\varSigma \cdot p}] = -i\hbar c [\boldsymbol{\alpha \cdot \nabla}, \boldsymbol{\varSigma \cdot p}] = -i\hbar c \left( \begin{matrix}
		0 & [\boldsymbol{\sigma \cdot \nabla}, \boldsymbol{\sigma \cdot p}] \\
		[\boldsymbol{\sigma \cdot \nabla}, \boldsymbol{\sigma \cdot p}] & 0
	\end{matrix} \right)
\]
\end{proof}

显然有:
\[
	[\boldsymbol{\sigma \cdot \nabla}, \boldsymbol{\sigma \cdot p}] = -\frac{1}{i\hbar} [\boldsymbol{\sigma \cdot p}, \boldsymbol{\sigma \cdot p}] = 0.
\]

则有:
\[
	[H, \boldsymbol{\varSigma \cdot p}] = 0.
\]

\end{enumerate}

\end{document}