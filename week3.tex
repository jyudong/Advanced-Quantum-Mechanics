\documentclass[reqno,a4paper,12pt]{amsart}

\usepackage{amsmath,amssymb,amsthm,geometry,xcolor,soul,graphicx}
\usepackage{titlesec}
\usepackage{enumerate}
\usepackage{lipsum}
\usepackage{listings}
%\usepackage{mathtools}
\RequirePackage[most]{tcolorbox}
\usepackage{braket}
\allowdisplaybreaks[4] %align公式跨页
\usepackage{xeCJK}
\setCJKmainfont{Kai}
\geometry{left=0.7in, right=0.7in, top=1in, bottom=1in}

\renewcommand{\baselinestretch}{1.3}

\title{高等量子力学第三次作业}
\author{董建宇 ~~ 202328000807038}

\begin{document}

\maketitle
\titleformat{\section}[hang]{\small}{\thesection}{0.8em}{}{}
\titleformat{\subsection}[hang]{\small}{\thesubsection}{0.8em}{}{}

\textbf{习题1.21}

在相干态表象下求$\hat{a}$和$\hat{a}^\dagger\hat{a}$的本征值和本征态。
\begin{tcolorbox}[breakable, colback = black!5!white, colframe = black]
可以计算:
\[
	\hat{a} \ket{\bar{\alpha}} = \sum_{n=1}^{\infty} \frac{\alpha^n}{\sqrt{n!}} \hat{a} \ket{n} + \hat{a} \ket{0} = \alpha \sum_{n=1}^\infty \frac{\alpha^{n-1}}{\sqrt{(n-1)!}} \ket{n-1} = \alpha \sum_{n=0}^\infty \frac{\alpha^n}{\sqrt{n!}} \ket{n} = \alpha \ket{\bar{\alpha}}.
\]
将$\hat{a}^\dagger\hat{a}$本征态记作$\ket{\psi}$,本征值记作$\lambda$,在相干态表象下本征值方程为:
\[
	\bra{\bar{\alpha}} \hat{a}^\dagger\hat{a} \ket{\psi} = \alpha^* \frac{\partial}{\partial \alpha^*} \bra{\bar{\alpha}} \psi \rangle = \lambda \bra{\bar{\alpha}} \psi \rangle.
\]
不难注意到,当$\bra{\bar{\alpha}} \psi \rangle = c_n (\alpha^*)^{n}$时,上述本征值方程为:
\[
	\alpha^* \frac{\partial}{\partial \alpha^*} c_n (\alpha^*)^{n} = nc_n (\alpha^*)^{n} = \lambda c_n (\alpha^*)^{n}.
\]
即$\lambda = n$,由归一化条件$\bra{\psi}\psi\rangle = \int d\mu(\alpha) \bra{\psi} \bar{\alpha}\rangle \bra{\bar{\alpha}} \psi\rangle = 1$可以计算$c_n = \frac{1}{\sqrt{n!}}$,即$\hat{a}^\dagger\hat{a}$的本征态$\ket{\psi}$满足$\bra{\bar{\alpha}} \psi \rangle = \frac{(\alpha^*)^n}{\sqrt{n!}}$本征值为$n$。

注意到$\bra{\bar{\alpha}} = \sum_{n} \frac{(\alpha^*)^n}{\sqrt{n!}}\bra{n}$,则$\hat{a}^\dagger\hat{a}$的本征态为$\ket{\psi} = \ket{n}$。

\end{tcolorbox}

\textbf{习题1.22}

证明任意密度矩阵算符满足
\[
	\mathbf{Tr}(\hat{\rho}^2) \leq 1.
\]

当且仅当
\[
	\mathbf{Tr}(\hat{\rho}^2) = 1.
\]

体系处于纯态。
\begin{tcolorbox}[breakable, colback = black!5!white, colframe = black]
对于任意密度矩阵算符$\hat{\rho} = \sum_i p_i \ket{\psi_i} \bra{\psi_i}$,可以计算:
\[
	\hat{\rho}^2 = \sum_{i,j} p_i p_j \ket{\psi_i} \bra{\psi_i} \psi_j \rangle \bra{\psi_j}.
\]
则可以计算:
\[
	\mathbf{Tr}(\hat{\rho}^2) = \sum_{i,j} p_i p_j \bra{\psi_j} \psi_i \rangle \bra{\psi_i} \psi_j \rangle = \sum_{i,j} p_i p_j \vert \bra{\psi_i} \psi_j \rangle \vert^2 \leq \sum_{i,j} p_i p_j = \left(\sum_i p_i\right) \left( \sum_j p_j \right) = 1.
\]
对于纯态,密度矩阵算符可以写作$\hat{\rho} = \ket{\psi}\bra{\psi}$,则容易计算:
\[
	\mathbf{Tr}(\hat{\rho}^2) = \mathbf{Tr}(\hat{\rho}) = 1.
\]
对于任意密度矩阵算符$\hat{\rho} = \sum_i p_i \ket{\psi_i}\bra{\psi_i}$,因为其为厄米算符,所以总可以选取一个表象,使得密度矩阵算符对角化,记作$\hat{\rho} = \sum_j p_j' \ket{\phi_j}\bra{\phi_j}$,其中$\bra{\phi_i} \phi_j \rangle = \delta_{ij}$。则可以计算:
\[
	\mathbf{Tr}(\hat{\rho}^2) = \sum_{i,j}p_i'p_j' \vert \bra{\phi_i} \phi_j \rangle \vert^2 = \sum_{i}p_i'^2 \leq \left( \sum_i p_i' \right)^2 = 1.
\]
等号成立的条件为只存在唯一一个$p' = 1$。即此时密度矩阵算符为:
\[
	\hat{\rho} = \ket{\phi} \bra{\phi}.
\]
即体系处于纯态。

综上所述,密度矩阵算符满足$\mathbf{Tr}(\hat{\rho}^2) \leq 1$,当且仅当$\mathbf{Tr}(\hat{\rho}^2) = 1$时体系处于纯态。
\end{tcolorbox}

\textbf{习题1.23}

证明二能级系统的密度算符矩阵可以写作
\[
	\hat{\rho} = \frac{\mathbf{I} + \vec{r} \cdot \vec{\sigma}}{2}
\]

其中$\vec{\sigma} = \sigma_x \vec{e}_x + \sigma_y \vec{e}_y + \sigma_z \vec{e}_z$,$\sigma_x,\sigma_y,\sigma_z$为Pauli矩阵,$\vec{e}_x,\vec{e}_y,\vec{e}_z$为$x, y, z$方向的单位矢量。矢量$\vec{r}$满足:
\[
	\vert \vec{r} \vert = r \leq 1.
\]

当且仅当$r=1$时,体系处于纯态。
\begin{tcolorbox}[breakable, colback = black!5!white, colframe = black]
二能级系统的密度算符矩阵可以写作:
\[
	\hat{\rho} = p \ket{\psi_1}\bra{\psi_1} + (1-p)\ket{\psi_2}\bra{\psi_2}.
\]
考虑这个算符在$\{ \ket{u_1}, \ket{u_2} \}$表象下的矩阵元可以计算得:
\begin{align*}
	\bra{u_1} \hat{\rho} \ket{u_1} =& p\alpha_1\alpha_1^* + (1-p) \beta_1\beta_1^*; \\
	\bra{u_1} \hat{\rho} \ket{u_2} =& p\alpha_1\alpha_2^* + (1-p) \beta_1\beta_2^*; \\
	\bra{u_2} \hat{\rho} \ket{u_1} =& p\alpha_2\alpha_1^* + (1-p) \beta_2\beta_1^*; \\
	\bra{u_2} \hat{\rho} \ket{u_2} =& p\alpha_2\alpha_2^* + (1-p) \beta_2\beta_2^*.
\end{align*}
其中:
\[
	\alpha_1 = \bra{u_1} \psi_1 \rangle, ~ \alpha_2 = \bra{u_2} \psi_1 \rangle, ~ \beta_1 = \bra{u_1} \psi_2 \rangle, ~ \beta_2 = \bra{u_2} \psi_2 \rangle.
\]
则不难注意到:
\[
	\vert \alpha_1 \vert^2 + \vert \alpha_2 \vert^2 = 1, ~ \vert \beta_1 \vert^2 + \vert \beta_2 \vert^2 = 1, ~ \vert \alpha_1 \vert^2 + \vert \beta_1 \vert^2 = 1, ~ \vert \alpha_2 \vert^2 + \vert \beta_2 \vert^2 = 1.
\]
可以计算:
\[
	\bra{u_1} \hat{\rho} \ket{u_1} + \bra{u_2} \hat{\rho} \ket{u_2} = 1.
\]
同时,令:
\begin{align*}
	x =& \bra{u_1} \hat{\rho} \ket{u_2} + \bra{u_2} \hat{\rho} \ket{u_1} = p(\alpha_1\alpha_2^*+ \alpha_2\alpha_1^*) + (1-p) (\beta_1\beta_2^* + \beta_2\beta_1^*); \\
	iy =& \bra{u_1} \hat{\rho} \ket{u_2} - \bra{u_2} \hat{\rho} \ket{u_1} = p(\alpha_1\alpha_2^* - \alpha_2\alpha_1^*) + (1-p) (\beta_1\beta_2^* - \beta_2\beta_1^*); \\
	z =& \bra{u_1} \hat{\rho} \ket{u_1} - \bra{u_2} \hat{\rho} \ket{u_2} = p(\alpha_1\alpha_1^* - \alpha_2\alpha_2^*) + (1-p)(\beta_1\beta_1^* - \beta_2\beta_2^*).
\end{align*}
则密度矩阵算符的矩阵表示可以写作:
\[
	\rho = \left( \begin{matrix}
		\frac{1}{2}+\frac{z}{2} & \frac{x}{2}+\frac{iy}{2} \\
		\frac{x}{2}-\frac{iy}{2} & \frac{1}{2}-\frac{z}{2}
	\end{matrix} \right) = \frac{\mathbf{I} + \vec{r}\cdot\vec{\sigma}}{2}
\]
可以计算:
\begin{align*}
	\vert \vec{r} \vert^2 =& \vert x \vert^2 + \vert y \vert^2 + \vert z \vert^2 \\
	=& p^2 + (1-p)^2 + 2p(1-p) (4\mathbf{Re}(\alpha_1\alpha_2^*\beta_1^*\beta_2) - (1-2\vert\alpha_1\vert^2)^2) \\
	\leq& p^2 + (1-p)^2 + 2p(1-p) (4\vert \alpha_1 \vert^2 - 4\vert \alpha_1 \vert^4 - (1-2\vert\alpha_1\vert^2)^2) \\
	=& p^2 + (1-p)^2 + 2p(1-p) - 4p(1-p)(2\vert \alpha_1 \vert^2 - 1)^2 \\
	=& 1 - 4p(1-p)(2\vert \alpha_1 \vert^2 - 1)^2 \\
	\leq& 1.
\end{align*}
则有:
\[
	\vert \vec{r} \vert = r \leq 1.
\]
当$r = 1$时,有:$p = 0$或$p = 1$,显然此时有$\hat{\rho}^2 = \hat{\rho}$,体系处于纯态。

当体系处于纯态时,有$\hat{\rho}^2 = \hat{\rho}$,则可以计算:
\begin{align*}
	\hat{\rho}^2 =& \frac{1}{4} (\mathbf{I}+\vec{r}\cdot\vec{\sigma})^2 \\
	=& \frac{1}{4} (\mathbf{I} + 2\vec{r}\cdot\vec{\sigma} + (\vec{r}\cdot\vec{\sigma})^2) \\
	=& \frac{1}{4} ((1+r^2)\mathbf{I} + 2\vec{r}\cdot\vec{\sigma}) \\
	=& \hat{\rho}.
\end{align*}
则有$\vert\vec{r}\vert = r = 1$。

综上所述,二能级系统的密度算符矩阵可以写作:
\[
	\hat{\rho} = \frac{\mathbf{I} + \vec{r} \cdot \vec{\sigma}}{2}
\]
当且仅当$r=1$时,体系处于纯态。
\end{tcolorbox}

\textbf{习题1.24}

证明$\hat{\rho}_1 = \mathbf{Tr}_2(\ket{\psi}_{12}{ }_{12}\bra{\psi})$为密度矩阵算符,并且$\hat{\rho}_1$为纯态的充要条件为$\ket{\psi}_{12}$为直积态。
\begin{tcolorbox}[breakable, colback = black!5!white, colframe = black]
可以计算:
\[
	\mathbf{Tr}(\hat{\rho}_1) = \mathbf{Tr}_1(\mathbf{Tr}_2 (\ket{\psi}_{12} { }_{12}\bra{\psi})) = \mathbf{Tr}_{12} (\ket{\psi}_{12} { }_{12}\bra{\psi}) = 1.
\]
\[
	\hat{\rho}_1^\dagger = [\mathbf{Tr}_2(\ket{\psi}_{12} { }_{12}\bra{\psi})]^\dagger = \mathbf{Tr}_2(\ket{\psi}_{12} { }_{12}\bra{\psi}) = \hat{\rho}_1
\]
对于任意态矢量$\ket{a} = \sum \alpha_m \ket{m}$,可以计算:
\begin{align*}
	\bra{a} \hat{\rho}_1 \ket{a} =& \bra{a} \mathbf{Tr}_2(\ket{\psi}_{12} { }_{12}\bra{\psi}) \ket{a} \\
	=& \mathbf{Tr}_1(\ket{a}\bra{a} \mathbf{Tr}_2(\ket{\psi}_{12} { }_{12}\bra{\psi}) ) \\
	=& \mathbf{Tr}_{12} (\ket{a}\bra{a}\otimes\mathbf{1}_2) (\ket{\psi}_{12} { }_{12}\bra{\psi}) \\
	=& { }_{12}\bra{\psi} (\ket{a}\bra{a}\otimes \mathbf{1}_2) \ket{\psi}_{12} \\
	=& \sum_{m,n}c_{mn}^* { }_1\bra{m}\otimes { }_2\bra{n} (\ket{a}\bra{a}\otimes \mathbf{1}_2) \sum_{m',n'}c_{m'n'} \ket{m'}_1 \otimes \ket{n'}_2 \\
	=& \sum_{m,m',n} c_{mn}^*c_{m'n} \bra{m} a \rangle \bra{a} m' \rangle \\
	=& \sum_n \left( \sum_{m} c_{mn}^* \bra{m} a\rangle \right) \left( \sum_{m} c_{mn} \bra{a} m \rangle \right) \\
	\geq& 0.
\end{align*}
若$\ket{\psi}_{12} = \ket{\phi}_1 \otimes \ket{\chi}_2$,则有:
\[
	\hat{\rho}_1 = \mathbf{Tr}_2(\ket{\psi}_{12} { }_{12}\bra{\psi}) = { }_2\bra{\chi} \chi \rangle_2 \ket{\phi}_1 { }_1 \bra{\phi} = \ket{\phi}_1 { }_1 \bra{\phi} = \ket{\phi} \bra{\phi}.
\]
则有:
\[
	\hat{\rho}_1^2 =  \ket{\phi} \bra{\phi} \phi \rangle \bra{\phi} =  \ket{\phi} \bra{\phi} = \hat{\rho}_1.
\]
若$\hat{\rho}_1$为纯态,即$\hat{\rho}_1^2 = \hat{\rho}_1$。可以计算:
\begin{align*}
	\hat{\rho}_1 =& \mathbf{Tr}_2 (\sum_{m,n} c_{mn} \ket{m}\otimes \ket{n} \sum_{m',n'} c_{m'n'}^* \bra{m'} \bra{n'}) \\
	=& \sum_{m,n,m',n'} c_{mn} c_{m'n'}^* \ket{m} \bra{m'} \bra{n'} n \rangle \\
	=& \sum_{m,m',n} c_{mn} c_{m',n}^* \ket{m} \bra{m'} \\
	\hat{\rho}_1^2 =& \sum_{m_1,m_2,m_3,m_4,n_1,n_2} c_{m_1,n_1}c_{m_2,n_1}^* c_{m_3,n_2} c_{m_4,n_2}^* \ket{m_1} \bra{m_2} m_3 \rangle \bra{m_4} \\
	=& \sum_{m,m',n} c_{mn} c_{m'n}^* \ket{m} \bra{m'} = \hat{\rho}_1
\end{align*}
当且仅当满足$c_{mn} = \alpha_m \beta_n$时上式成立。此时可以计算:
\begin{align*}
	\hat{\rho}_1 =& \mathbf{Tr}_2 (\sum_{m,n} \alpha_m\beta_n \ket{m}\otimes \ket{n} \sum_{m',n'} \alpha_{m'}^*\beta_{n'}^* \bra{m'} \bra{n'}) \\
	=& \sum_{m,n,m',n'} \alpha_m\beta_n \alpha_m^*\beta_n^* \ket{m} \bra{m'} \bra{n'} n \rangle \\
	=& \sum_{m,m'} \alpha_m \alpha_{m'}^* \ket{m} \bra{m'} \\
	\hat{\rho}_1^2 =& \sum_{m_1,m_2,m_3,m_4} \alpha_{m_1}\alpha_{m_2}^* \alpha_{m_3} \alpha_{m_4}^* \ket{m_1} \bra{m_2} m_3 \rangle \bra{m_4} \\
	=& \sum_{m,m'} \alpha_m \alpha_{m'}^* \ket{m} \bra{m'}.
\end{align*}
当$c_{mn} = \alpha_m\beta_n$时,则有:
\[
	\ket{\psi} = \sum_{mn} \alpha_m \beta_n \ket{m} \otimes \ket{n} = \left( \sum_m \alpha_m \ket{m} \right) \otimes \left( \sum_{n} \beta_n \ket{n} \right).
\]
即$\ket{\psi}_{12}$可以写作直积态。

综上所述,$\hat{\rho}_1$为纯态的充要条件为$\ket{\psi}_{12}$为直积态。
\end{tcolorbox}

\textbf{习题1.25}

证明对于量子态
\[
	\ket{B} = \frac{\ket{00}+\ket{11}}{\sqrt{2}}
\]

可以选择$\vec{n}_1, \vec{n}_2, \vec{n}_3, \vec{n}_4$,使得$\vert \langle \hat{B} \rangle \vert = 2\sqrt{2}$
\begin{tcolorbox}[breakable, colback = black!5!white, colframe = black]
选取四个向量分别为:
\begin{align*}
	\vec{n}_1 =& (1/\sqrt{2} \ \ \ 0 \ \ \ 1/\sqrt{2});& & \vec{n}_2 = (1 \ \ \ 0 \ \ \ 0); \\
	\vec{n}_3 =& (1/\sqrt{2} \ \ \ 0 \ \ \ -1/\sqrt{2});& & \vec{n}_4 = (0 \ \ \ 0 \ \ \ 1).
\end{align*}
可以计算:
\begin{align*}
	\vec{\sigma}_1 \cdot \vec{n}_1 =& \frac{1}{\sqrt{2}}\sigma_x+\frac{1}{\sqrt{2}}\sigma_z = \left( \begin{matrix}
		\frac{1}{\sqrt{2}} & \frac{1}{\sqrt{2}} \\
		\frac{1}{\sqrt{2}} & -\frac{1}{\sqrt{2}}
	\end{matrix} \right); \\
	\vec{\sigma}_2 \cdot (\vec{n}_2+\vec{n}_4) =& \sigma_x+\sigma_z = \left( \begin{matrix}
		1 & 1 \\
		1 & -1
	\end{matrix} \right); \\
	\vec{\sigma}_1 \cdot \vec{n}_3 =& \frac{1}{\sqrt{2}}\sigma_x-\frac{1}{\sqrt{2}}\sigma_z = \left( \begin{matrix}
		-\frac{1}{\sqrt{2}} & \frac{1}{\sqrt{2}} \\
		\frac{1}{\sqrt{2}} & \frac{1}{\sqrt{2}}
	\end{matrix} \right); \\
	\vec{\sigma}_2 \cdot (\vec{n}_2-\vec{n}_4) =& \sigma_x-\sigma_z = \left( \begin{matrix}
		1 & 1 \\
		1 & -1
	\end{matrix} \right).
\end{align*}
则可以计算:
\[
	\hat{B} = \left( \begin{matrix}
		\frac{1}{\sqrt{2}} & \frac{1}{\sqrt{2}} & \frac{1}{\sqrt{2}} & \frac{1}{\sqrt{2}} \\
		\frac{1}{\sqrt{2}} & -\frac{1}{\sqrt{2}} & \frac{1}{\sqrt{2}} & -\frac{1}{\sqrt{2}} \\
		\frac{1}{\sqrt{2}} & \frac{1}{\sqrt{2}} & -\frac{1}{\sqrt{2}} & -\frac{1}{\sqrt{2}} \\
		\frac{1}{\sqrt{2}} & -\frac{1}{\sqrt{2}} & -\frac{1}{\sqrt{2}} & \frac{1}{\sqrt{2}}
	\end{matrix} \right) 
	+ 
	\left( \begin{matrix}
		\frac{1}{\sqrt{2}} & -\frac{1}{\sqrt{2}} & -\frac{1}{\sqrt{2}} & \frac{1}{\sqrt{2}} \\
		-\frac{1}{\sqrt{2}} & -\frac{1}{\sqrt{2}} & \frac{1}{\sqrt{2}} & \frac{1}{\sqrt{2}} \\
		-\frac{1}{\sqrt{2}} & \frac{1}{\sqrt{2}} & -\frac{1}{\sqrt{2}} & \frac{1}{\sqrt{2}} \\
		\frac{1}{\sqrt{2}} & \frac{1}{\sqrt{2}} & \frac{1}{\sqrt{2}} & \frac{1}{\sqrt{2}}
	\end{matrix} \right) 
	= 
	\left( \begin{matrix}
		\sqrt{2} & 0 & 0 & \sqrt{2} \\
		0 & -\sqrt{2} & \sqrt{2} & 0 \\
		0 & \sqrt{2} & -\sqrt{2} & 0 \\
		\sqrt{2} & 0 & 0 & \sqrt{2}
	\end{matrix} \right).
\]
则可以计算:
\[
	\bra{B} \hat{B} \ket{B} = \left( \begin{matrix}
		\frac{1}{\sqrt{2}} & 0 & 0 & \frac{1}{\sqrt{2}}
	\end{matrix} \right)
	\left( \begin{matrix}
		\sqrt{2} & 0 & 0 & \sqrt{2} \\
		0 & -\sqrt{2} & \sqrt{2} & 0 \\
		0 & \sqrt{2} & -\sqrt{2} & 0 \\
		\sqrt{2} & 0 & 0 & \sqrt{2}
	\end{matrix} \right)
	\left( \begin{matrix}
		\frac{1}{\sqrt{2}} \\
		0 \\
		0 \\
		\frac{1}{\sqrt{2}}
	\end{matrix} \right) = 2\sqrt{2}.
\]
\end{tcolorbox}

\textbf{习题1.26}

证明
\[
	\hat{\rho} = \frac{e^{-\beta \hat{H}}}{\mathbf{Tr}(e^{-\beta \hat{H}})}
\]

为密度算符。其中$\beta = k_B T$,$k_B$为玻尔兹曼常数,$T$为温度。
\begin{tcolorbox}[breakable, colback = black!5!white, colframe = black]
因为$\hat{H}^\dagger = \hat{H}$,则有:
\[
	\hat{\rho}^\dagger = \frac{1}{\mathbf{Tr}(e^{-\beta \hat{H}})}\left( \sum_{n=0}^\infty \frac{(-\beta\hat{H})^n}{n!} \right)^\dagger = \frac{1}{\mathbf{Tr}(e^{-\beta \hat{H}})}\left( \sum_{n=0}^\infty \frac{(-\beta\hat{H})^n}{n!} \right) = \frac{e^{-\beta \hat{H}}}{\mathbf{Tr}(e^{-\beta \hat{H}})} = \hat{\rho}.
\]

可以计算:
\[
	\mathbf{Tr}(\hat{\rho}) = \frac{\mathbf{Tr}(e^{-\beta \hat{H}})}{\mathbf{Tr}(e^{-\beta \hat{H}})} = 1.
\]

由于指数函数的正定性,即对于任意量子态$\ket{\psi}$,都有:
\[
	\bra{\psi} e^{-\beta \hat{H}} \ket{\psi} > 0.
\]

则有$\hat{\rho} \geq 0$.

综上所述,$\hat{\rho} = \frac{e^{-\beta \hat{H}}}{\mathbf{Tr}(e^{-\beta \hat{H}})}$为密度算符。
\end{tcolorbox}

\textbf{习题1.27}

证明下列关系式:
\[
	e^{\hat{A}} \hat{B} e^{-\hat{A}} = \hat{B} + \frac{1}{1!}[\hat{A}, \hat{B}] + \frac{1}{2!}[\hat{A},[\hat{A},\hat{B}]] + \cdots
\]

如果算符$\hat{A}$存在对角化
\[
	\hat{A} = \sum_a \ket{a} a \bra{a}
\]

其中$\bra{a} a' \rangle = \delta_{a,a'}$,那么我们定义对任意函数$f(x)$
\[
	f(\hat{A}) = \sum_a \ket{a}f(a)\bra{a}
\]

可以证明当函数$f(x)$可泰勒展开时,上述定义与定义(1.210)一致。

对于在区域$D$上解析的函数$f(z)$,如果算符$\hat{A}$的本征值在$D$内,那么算符函数
\[
	f(\hat{A}) = \frac{1}{2\pi i} \int_{\varGamma} dz f(z) \frac{1}{z-A}
\]
其中$\varGamma$为区域$D$中包含所有$\hat{A}$的本征值的任意正向闭合路径。可以证明该定义与定义(1.225)的一致性。
\[
	(\hat{A}\hat{B})^{-1} = \hat{B}^{-1} \hat{A}^{-1}, \ \ (\hat{A}+\hat{B})^{-1} = \hat{A}^{-1} - \hat{A}^{-1}\hat{B}(\hat{A}+\hat{B})^{-1}
\]
\begin{tcolorbox}[breakable, colback = black!5!white, colframe = black]
对于$e^{\hat{A}}$进行Taylor展开有:
\[
	e^{\hat{A}} = \sum_{n=0}^\infty \frac{1}{n!} \hat{A}^n = \left( 1 + \hat{A} + \frac{1}{2!}\hat{A}^2 + \cdots \right).
\]
则展开$e^{\hat{A}} \hat{B} e^{-\hat{A}}$,可得:
\[
	e^{\hat{A}} \hat{B} e^{-\hat{A}} = \left( 1 + \hat{A} + \frac{1}{2!}\hat{A}^2 + \cdots \right) \hat{B} \left( 1 - \hat{A} + \frac{1}{2!}\hat{A}^2 + \cdots \right).
\]
其中,可以计算$\hat{A}$的零阶项为$\hat{B}$,$\hat{A}$的一阶项为$\hat{A}\hat{B} - \hat{B}\hat{A} = [\hat{A}, \hat{B}]$,$\hat{A}$的二阶项为$\frac{1}{2!}\hat{A}^2\hat{B}-\hat{A}\hat{B}\hat{A} + \frac{1}{2!}\hat{B}\hat{A}^2 = \frac{1}{2!}[\hat{A}, [\hat{A},\hat{B}]]$,以此类推,可得:
\[
	e^{\hat{A}} \hat{B} e^{-\hat{A}} = \hat{B} + \frac{1}{1!}[\hat{A}, \hat{B}] + \frac{1}{2!}[\hat{A},[\hat{A},\hat{B}]] + \cdots
\]
可以计算:
\[
	\hat{A}^n = \sum_{a_1,a_2,...,a_n} \prod_{i=1}^n a_i \ket{a_1} \bra{a_1} a_2 \rangle \bra{a_2} \cdots \ket{a_n} \bra{a_n} = \sum_a a^n \ket{a} \bra{a}.
\]
则有:
\begin{align*}
	f(\hat{A}) =& \sum_{n=0}^\infty \frac{1}{n!} f^{(n)}(0) \hat{A}^n = \sum_{n=0}^\infty \frac{1}{n!} f^{(n)}(0) \sum_{a} a^n \ket{a} \bra{a} \\
	=& \sum_a \ket{a} \sum_{n=0}^\infty \frac{1}{n!} f^{(n)}(0) a^n \bra{a} = \sum_a \ket{a} f(a) \bra{a}.
\end{align*}
令$g(\hat{A}) = \frac{1}{z-\hat{A}}$,则有:
\[
	g(\hat{A}) = \sum_a \ket{a} g(a) \bra{a} = \sum_a \ket{a} \frac{1}{z-a} \bra{a}.
\]
从而可以计算算符函数:
\[
	f(\hat{A}) = \frac{1}{2\pi i} \int_\varGamma dz f(z) g(\hat{A}) = \frac{1}{2\pi i} \sum_a \ket{a} \int_\varGamma dz f(z) \frac{1}{z-a} \bra{a} = \sum_a \ket{a} f(a) \bra{a}.
\]
可以计算:
\[
	(\hat{A}\hat{B})^{-1}(\hat{A}\hat{B}) = \hat{B}^{-1}\hat{A}^{-1} \hat{A}\hat{B} = \hat{B}^{-1} \hat{B} = \mathbf{1}.
\]
\[
	(\hat{A}+\hat{B})^{-1}(\hat{A}+\hat{B}) = (\hat{A}^{-1}-\hat{A}^{-1} \hat{B} (\hat{A}+\hat{B})^{-1}) (\hat{A}+\hat{B}) = \mathbf{1} + \hat{A}^{-1}\hat{B} - \hat{A}^{-1}\hat{B} = \mathbf{1}.
\]
其中利用了$(\hat{A}+\hat{B})^{-1}(\hat{A}+\hat{B}) = \mathbf{1}$。

综上所述,有:
\[
	(\hat{A}\hat{B})^{-1} = \hat{B}^{-1} \hat{A}^{-1}, \ \ (\hat{A}+\hat{B})^{-1} = \hat{A}^{-1} - \hat{A}^{-1}\hat{B}(\hat{A}+\hat{B})^{-1}.
\]
\end{tcolorbox}
\end{document}

\begin{tcolorbox}[breakable, colback = black!5!white, colframe = black]

\end{tcolorbox}