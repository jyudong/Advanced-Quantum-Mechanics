\documentclass[reqno,a4paper,12pt]{amsart}

\usepackage{amsmath,amssymb,amsthm,geometry,xcolor,soul,graphicx}
\usepackage{titlesec}
\usepackage{enumerate}
%\usepackage{lipsum} used a paragraph to test the environment 
\usepackage{listings}
%\RequirePackage[most]{tcolorbox}
\usepackage{braket}
\allowdisplaybreaks[4] %align公式跨页
\usepackage{xeCJK}
\setCJKmainfont[AutoFakeBold = true]{Kai}
\geometry{left=0.7in, right=0.7in, top=1in, bottom=1in}

\renewcommand{\baselinestretch}{1.3}

\title{高等量子力学第九次作业}
\author{董建宇 ~~ 202328000807038}

%\setlength{\parindent}{2pt}

\begin{document}

\maketitle

\textbf{习题4.7}

求解球形方势阱中$l=0$时的束缚态。

\begin{proof}
由薛定谔方程,可得:
\[
	\left( -\frac{\hbar^2}{2m} \nabla^2 + V(r) \right) \varphi(\vec{r}) = E \varphi(\vec{r}).
\]

在球坐标系下,分离变量可得,方程的解可以写作:
\[
	\varphi_{l,m}(\vec{r}) = \frac{1}{r}u_l(r) Y_l^m(\theta, \phi).
\]

其中,$u_l(r)$满足:
\[
	\left( -\frac{\hbar^2}{2m} \frac{d^2}{dr^2} + \frac{l(l+1)\hbar^2}{2mr^2} + V(r) \right) u_l(r) = E u_l(r).
\]

对于球形方势阱,可以选取低势能面为零势能面,则势能函数写作:
\[
	V(r) = \left\{ \begin{aligned}
		&0, & & 0<r<r_0; \\
		&V_0, & & r>r_0.
	\end{aligned} \right.
\]

对于束缚态有:$0<E<V_0$,当$l=0$时,记$u(r) = u_0(r)$,$0<r<r_0$范围内有:
\[
	-\frac{\hbar^2}{2m} \frac{d^2}{dr^2} u(r) = E u(r).
\]

通解为:
\[
	u(r) = A\sin\left( \sqrt{\frac{2mE}{\hbar^2}} r \right) + B\cos\left( \sqrt{\frac{2mE}{\hbar^2}} r \right).
\]

利用边界条件$u(0) = 0$,可得$B=0$。

在$r>r_0$范围内有:
\[
	\left( -\frac{\hbar^2}{2m} \frac{d^2}{dr^2} + V_0 \right) u_l(r) = E u_l(r).
\]

通解为:
\[
	u(r) = C \exp\left( -\sqrt{\frac{2m(V_0-E)}{\hbar^2}} r \right) + D \exp\left( \sqrt{\frac{2m(V_0-E)}{\hbar^2}} r \right)
\]

利用边界条件$u(\infty) = 0$,可得$D=0$。

利用连续性条件
\[
	\lim_{r\to r_0^-} u(r) = \lim_{r\to r_0^+} u(r); \ \ \left. \frac{du(r)}{dr} \right\vert_{r\to r_0^-} = \left. \frac{du(r)}{dr} \right\vert_{r\to r_0^+}.
\]

可得:
\begin{align*}
	&A\sin\left( \sqrt{\frac{2mE}{\hbar^2}} r_0 \right) = C \exp \left( -\sqrt{\frac{2m(V_0-E)}{\hbar^2}} r_0 \right); \\
	&A\sqrt{\frac{2mE}{\hbar^2}} \cos\left( \sqrt{\frac{2mE}{\hbar^2}} r_0 \right) = -\sqrt{\frac{2m(V_0-E)}{\hbar^2}} C \exp \left( -\sqrt{\frac{2m(V_0-E)}{\hbar^2}} r_0 \right).
\end{align*}

若存在$A$和$C$,则系数行列式为$0$,记$k=\sqrt{\frac{2mE}{\hbar^2}}, k' = \sqrt{\frac{2m(V_0-E)}{\hbar^2}}$即
\[
	\left\vert \begin{matrix}
		\sin(kr_0) & -\exp(-k'r_0) \\
		k\cos(kr_0) & k'\exp(-k'r_0)
	\end{matrix} \right\vert = \exp(-k'r_0) (k'\sin(kr_0) + k\cos(kr_0)) = 0.
\]

即$l=0$时束缚态能量$E$满足如下方程:
\[
	\sqrt{\frac{E}{V_0-E}} = \tan\left(\sqrt{\frac{2mE}{\hbar^2}}r_0\right).
\]

波函数可以写作:
\[
	\varphi (r,\theta, \phi) = \frac{1}{r}u(r) Y_0^0(\theta, \phi).
\]

可以将波函数归一化得到束缚态在坐标表象下的归一化的波函数。
\end{proof}

\medskip

\textbf{习题4.8}

试求散射势
\[
	V(r) = \gamma\delta(r-a)
\]

下的$s$波散射相移和散射截面,其中$a>0$。

\begin{proof}
与上题类似,薛定谔方程经分离变量可得:
\[
	\left( \frac{d^2}{dr^2} + k^2 - 2m\gamma \delta(r-a) \right)u = 0.
\]

其中边界条件为:
\[
	u(0) = 0; \ \ \ u(\infty) \to \sin(kr+\delta_0).
\]

当$r\neq a$时,$V(r) = 0$,则通解可以写为:
\[
	u(r) = \left\{ \begin{aligned}
		&A\sin(kr) + B\cos(kr) = \sqrt{A^2+B^2} \sin(kr+\delta_0); && r>a \\
		&C\sin(kr); && r<a.
	\end{aligned} \right.
\]

考虑连续性条件,有:
\[
	u(a+) = u(a-); \ \ u'(a+)-u'(a-) = 2m\gamma u(a).
\]

即
\begin{align*}
	&C\sin(ka) = A\sin(ka)+B\cos(ka) \\
	&C(k\cos(ka)+2m\gamma\sin(ka)) = Ak\cos(ka) - Bk\sin(ka)
\end{align*}

从而可得:
\[
	\frac{B}{A} = -\frac{2m\gamma \sin^2(ka)}{k+m\gamma\sin(2ka)}
\]

当$ka<<1$时,$s$波散射为主要贡献,相移为:
\[
	\delta_0 = \arctan\frac{B}{A} = -\arctan \left(\frac{2m\gamma \sin^2(ka)}{k+m\gamma\sin(2ka)} \right).
\]

散射截面为:
\[
	\sigma_0 = \frac{4\pi}{k^2}\sin^2\delta_0 = \frac{4\pi}{k^2} \frac{\left(\frac{2m\gamma \sin^2(ka)}{k+m\gamma\sin(2ka)} \right)^2}{1+\left(\frac{2m\gamma \sin^2(ka)}{k+m\gamma\sin(2ka)} \right)^2} = \frac{4\pi}{k^2} \frac{\left(2m\gamma \sin^2(ka)\right)^2}{\left(2m\gamma \sin^2(ka)\right)^2 + \left( k+m\gamma\sin(2ka) \right)^2}.
\]
\end{proof}

\medskip

\textbf{习题5.1}

$2$个全同玻色粒子在$3$个单粒子模式情况下,写出该Hilbert空间的所有基矢。

\begin{proof}
对于波色子体系,每个单粒子模式可以容纳任意多粒子,则该Hilbert空间的基矢为:
\[
	\ket{2,0,0}, \ \ket{0,2,0}, \ \ket{0,0,2}; \ \ket{1,1,0}, \ \ket{1,0,1}, \ \ket{0,1,1}.
\]
\end{proof}

\medskip

\textbf{习题5.2}

将上题中的玻色粒子换成费米粒子,写出Hilbert空间的全部基矢。

\begin{proof}
对于费米子体系,每一个粒子模式只能有$0$或$1$个粒子占据,基矢可以写为:
\[
	\ket{1,1,0}, \ \ket{1,0,1}, \ \ket{0,1,1}.
\]
\end{proof}


\end{document}