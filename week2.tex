\documentclass[reqno,a4paper,12pt]{amsart}

\usepackage{amsmath,amssymb,amsthm,geometry,xcolor,soul,graphicx}
\usepackage{titlesec}
\usepackage{enumerate}
\usepackage{lipsum}
\usepackage{listings}
\RequirePackage[most]{tcolorbox}
\usepackage{braket}
\allowdisplaybreaks[4] %align公式跨页
\usepackage{xeCJK}
\setCJKmainfont{Kai}
\geometry{left=0.7in, right=0.7in, top=1in, bottom=1in}

\renewcommand{\baselinestretch}{1.3}

\title{高等量子力学第二次作业}
\author{董建宇 ~~ 202328000807038}

\begin{document}

\maketitle
\titleformat{\section}[hang]{\small}{\thesection}{0.8em}{}{}
\titleformat{\subsection}[hang]{\small}{\thesubsection}{0.8em}{}{}

\textbf{习题1.9}

计算角动量分量之间的对易关系,并证明
\[
	[\hat{\vec{L}}^2, \hat{L}_z] = 0.
\]
\begin{tcolorbox}[breakable, colback = black!5!white, colframe = black]
利用基本对易关系$[\hat{x}_i,\hat{p}_j] = i\hbar\delta_{ij}$。则有:
\begin{align*}
	[\hat{L}_x, \hat{L}_y] =& [\hat{y}\hat{p}_z-\hat{z}\hat{p}_y, \hat{z}\hat{p}_x-\hat{x}\hat{p}_z] = i\hbar(\hat{x}\hat{p}_y - \hat{y}\hat{p}_x) = i\hbar\hat{L}_z, \\
	[\hat{L}_y, \hat{L}_z] =& [\hat{z}\hat{p}_x-\hat{x}\hat{p}_z, \hat{x}\hat{p}_y-\hat{y}\hat{p}_x] = i\hbar(\hat{y}\hat{p}_z - \hat{z}\hat{p}_y) = i\hbar\hat{L}_x, \\
	[\hat{L}_z, \hat{L}_x] =& [\hat{x}\hat{p}_y-\hat{y}\hat{p}_x, \hat{y}\hat{p}_z-\hat{z}\hat{p}_y] = i\hbar(\hat{z}\hat{p}_x - \hat{x}\hat{p}_z) = i\hbar\hat{L}_y.
\end{align*}
利用$\hat{\vec{L}}^2 = \hat{L}_x^2 + \hat{L}_y^2 + \hat{L}_z^2$,可以计算:
\[
	[\hat{\vec{L}}^2, \hat{L}_z] = [\hat{L}_x^2 + \hat{L}_y^2, \hat{L}_z] = -i\hbar(\hat{L}_x\hat{L}_y+\hat{L}_y\hat{L}_x) + i\hbar(\hat{L}_y\hat{L}_x+\hat{L}_x\hat{L}_y) = 0.
\]
\end{tcolorbox}

\textbf{习题1.10}

用Schwartz不等式证明不确定关系。
\begin{tcolorbox}[breakable, colback = black!5!white, colframe = black]
%对于任意一个量子态$\ket{\phi}$与两个可观测量$\hat{A},\hat{B}$,考虑:
%\begin{align*}
%	\ket{\Phi} =& (\hat{A}-\bra{\phi}\hat{A}\ket{\phi})\ket{\phi}; \\
%	\ket{\Psi} =& i(\hat{B}-\bra{\phi}\hat{B}\ket{\phi})\ket{\phi}.
%\end{align*}
%可以计算得:
%\begin{align*}
%	\bra{\Phi} \Phi \rangle =& \bra{\phi} (\hat{A} - \bra{\phi}\hat{A}\ket{\phi})^2 \ket{\phi}; \\
%	\bra{\Psi} \Psi \rangle =& \bra{\phi} (\hat{B} - \bra{\phi}\hat{B}\ket{\phi})^2 \ket{\phi}; \\
%	\bra{\Phi} \Psi \rangle =& \bra{\phi} (\hat{A}-\bra{\phi}\hat{A}\ket{\phi})i(\hat{B}-\bra{\phi}\hat{B}\ket{\phi}) \ket{\phi}
%\end{align*}
令$\hat{A}' = \hat{A} - \bra{\psi}\hat{A}\ket{\psi}$,$\hat{B}' = \hat{B} - \bra{\psi}\hat{B}\ket{\psi}$,则$\hat{A}'$和$\hat{B}'$为厄米算符,根据Schwartz不等式,有:
\[
	\bra{\psi}\hat{A}'^2\ket{\psi}\bra{\psi}\hat{B}'^2\ket{\psi} = (\bra{\psi} \hat{A}'^\dagger)(\hat{A}'\ket{\psi}) (\ket{\psi}\hat{B}'^\dagger)(\hat{B}'\ket{\psi}) \geq \left\vert \bra{\psi} \hat{A}'^\dagger \hat{B}' \ket{\psi} \right\vert^2 = \left\vert \bra{\psi} \hat{A}' \hat{B}' \ket{\psi} \right\vert^2
\]
记$\bra{\psi} \hat{A}' \hat{B}' \ket{\psi} = a+ib$,其中$a,b$为实数。则有:
\[
	\left\vert \bra{\psi} \hat{A}' \hat{B}' \ket{\psi} \right\vert^2 = a^2+b^2 = \frac{1}{4}\left( \left\vert \bra{\psi} [\hat{A}', \hat{B}'] \ket{\psi} \right\vert^2 + \left\vert \bra{\psi} \{\hat{A}', \hat{B}'\} \ket{\psi} \right\vert^2 \right) \geq \frac{1}{4} \left\vert \bra{\psi} [\hat{A}', \hat{B}'] \ket{\psi} \right\vert^2.
\]
则有:
\[
	\bra{\psi}\hat{A}'^2\ket{\psi}\bra{\psi}\hat{B}'^2\ket{\psi} \geq \frac{1}{4} \left\vert \bra{\psi} [\hat{A}', \hat{B}'] \ket{\psi} \right\vert^2.
\]
带入$\hat{A}' = \hat{A} - \bra{\psi}\hat{A}\ket{\psi}$,$\hat{B}' = \hat{B} - \bra{\psi}\hat{B}\ket{\psi}$,可得:
\[
	\bra{\psi}\hat{A}'^2\ket{\psi}\bra{\psi}\hat{B}'^2\ket{\psi} = (\Delta A)^2 (\Delta B)^2 \geq \frac{1}{4} \left\vert \bra{\psi} [\hat{A}', \hat{B}'] \ket{\psi} \right\vert^2 = \frac{1}{4} \left\vert \bra{\psi} i[\hat{A}', \hat{B}'] \ket{\psi} \right\vert^2.
\]
两侧开平方可得不确定关系。
\end{tcolorbox}

\textbf{习题1.11}

一个$\frac{1}{2}$自旋的希尔伯特空间的基矢记为$\{\ket{0},\ket{1}\}$,其上定义的两个厄米算符为:
\begin{align*}
	\hat{\sigma}_z = \ket{0}\bra{0} - \ket{1}\bra{1} \\
	\hat{\sigma}_x = \ket{0}\bra{1} + \ket{1}\bra{0}
\end{align*}

对于两个$\frac{1}{2}$自旋,其希尔伯特空间的基矢为$\{\ket{0}\otimes\ket{0}, \ket{0}\otimes\ket{1}, \ket{1}\otimes\ket{0}, \ket{1}\otimes\ket{1}\}$。在此希尔伯特空间上试证明
\[
	\{ \hat{\sigma}_z\otimes\hat{\sigma}_z, \hat{\sigma}_x\otimes\hat{\sigma}_x \}
\]

构成该希尔伯特空间上的厄米算符完备组。
\begin{tcolorbox}[breakable, colback = black!5!white, colframe = black]
在该基矢表象下,俩个算符的矩阵表示为:
\[
	\sigma_z\otimes\sigma_z = 
	\left(
	\begin{matrix}
		1 & 0 & 0 & 0 \\
		0 & -1 & 0 & 0 \\
		0 & 0 & -1 & 0 \\
		0 & 0 & 0 & 1
	\end{matrix}
	\right), ~~
	\sigma_x\otimes\sigma_x = 
	\left(
	\begin{matrix}
		0 & 0 & 0 & 1 \\
		0 & 0 & 1 & 0 \\
		0 & 1 & 0 & 0 \\
		1 & 0 & 0 & 0
	\end{matrix}
	\right)
\]

显然有:$[\hat{\sigma}_z\otimes\hat{\sigma}_z, \hat{\sigma}_x\otimes\hat{\sigma}_x] = 0$。

易知,$\sigma_z\otimes\sigma_z$具有两个本征值$1,-1$。

对应本征值为$1$的本征向量分别为:
\[
	u_1 = (\begin{matrix}
		1 & 0 & 0 & 0
	\end{matrix})^T, ~~
	u_2 = (\begin{matrix}
		0 & 0 & 0 & 1
	\end{matrix})^T.
\]
构造两组线性无关向量$u_1', ~u_2'$为:
\[
	u_1' = \frac{u_1+u_2}{\sqrt{2}} = (\begin{matrix}
		\frac{1}{\sqrt{2}} & 0 & 0 & \frac{1}{\sqrt{2}}
	\end{matrix})^T, ~~
	u_2' = \frac{u_1-u_2}{\sqrt{2}} = (\begin{matrix}
		\frac{1}{\sqrt{2}} & 0 & 0 & -\frac{1}{\sqrt{2}}
	\end{matrix})^T.
\]
可以计算:
\[
	\sigma_x\otimes\sigma_x u_1' = u_1', ~~ \sigma_x\otimes\sigma_x u_2' = -u_2'.
\]
即$u_1'$为$\sigma_x\otimes\sigma_x$本征值为$1$的本征向量,$u_2'$为$\sigma_x\otimes\sigma_x$本征值为$-1$的本征向量。

对应本征值为$-1$的本征向量分别为:
\[
	v_1 = (\begin{matrix}
		0 & 1 & 0 & 0
	\end{matrix})^T, ~~
	v_2 = (\begin{matrix}
		0 & 0 & 1 & 0
	\end{matrix})^T.
\]
同样地,构造两组线性无关向量$v_1',~v_2'$为:
\[
	v_1' = \frac{v_1+v_2}{\sqrt{2}} = (\begin{matrix}
		0 & \frac{1}{\sqrt{2}} & \frac{1}{\sqrt{2}} & 0
	\end{matrix})^T, ~~
	v_2' = \frac{v_1-v_2}{\sqrt{2}} = (\begin{matrix}
		0 & \frac{1}{\sqrt{2}} & -\frac{1}{\sqrt{2}} & 0
	\end{matrix})^T.
\]
可以计算:
\[
	\sigma_x\otimes\sigma_x v_1' = v_1', ~~ \sigma_x\otimes\sigma_x v_2' = -v_2'.
\]
即$v_1'$为$\sigma_x\otimes\sigma_x$本征值为$1$的本征向量,$v_2'$为$\sigma_x\otimes\sigma_x$本征值为$-1$的本征向量。

综上所述,$\{ \hat{\sigma}_z\otimes\hat{\sigma}_z, ~ \hat{\sigma}_x\otimes\hat{\sigma}_x \}$构成该希尔伯特空间上的厄米算符完备组。
\end{tcolorbox}

\textbf{习题1.12}

试证明:希尔伯特空间上的线性算符$\hat{O}$满足$\hat{O}^\dagger\hat{O} = \hat{O}\hat{O}^\dagger$当且仅当算符$\hat{O}$可以分解成$\hat{O}=\hat{A}+i\hat{B}$,其中算符$\hat{A}$和$\hat{B}$是互相对易的厄米算符。
\begin{tcolorbox}[breakable, colback = black!5!white, colframe = black]
当算符$\hat{O}$可以被分解成$\hat{O} = \hat{A}+i\hat{B}$,且$\hat{A}$和$\hat{B}$是对易的厄米算符,则有:
\[
	[\hat{O}, \hat{O}^\dagger] = [\hat{A}+i\hat{B}, \hat{A}-i\hat{B}] = 0.
\]
即$\hat{O}^\dagger\hat{O} = \hat{O}\hat{O}^\dagger$。

任意一个线性算符$\hat{O}$都可以分解为$\hat{O} = \hat{A} + i\hat{B}$,其中$\hat{A}, ~ \hat{B}$为厄米算符。当算符$\hat{O}$满足$\hat{O}^\dagger\hat{O} = \hat{O}\hat{O}^\dagger$时,有:
\[
	[\hat{O}, \hat{O}^\dagger] = [\hat{A}+i\hat{B}, \hat{A}-i\hat{B}] = -i[\hat{A}, \hat{B}] + i[\hat{B}, \hat{A}] = -2i[\hat{A}, \hat{B}] = 0.
\]
即$[\hat{A}, \hat{B}] = 0$。

综上所述,希尔伯特空间上的线性算符$\hat{O}$满足$\hat{O}^\dagger\hat{O} = \hat{O}\hat{O}^\dagger$当且仅当算符$\hat{O}$可以分解成$\hat{O}=\hat{A}+i\hat{B}$,其中算符$\hat{A}$和$\hat{B}$是互相对易的厄米算符。
\end{tcolorbox}

\textbf{习题1.13}

利用幺正变换重新推导
\[
	\bra{b} \hat{O} \ket{b'} = \sum_{a,a'} \bra{b} a \rangle \bra{a} \hat{O} \ket{a'} \bra{a'} b' \rangle.
\]
\begin{tcolorbox}[breakable, colback = black!5!white, colframe = black]
利用幺正变换
\[
	\ket{b} = \hat{U} \ket{a}, ~ \hat{U} = \sum \ket{b}\bra{a}.
\]
可以计算:
\begin{align*}
	\bra{b} \hat{O} \ket{b'} =& \bra{b} \hat{U}^\dagger \hat{U} \hat{O} \hat{U}^\dagger\hat{U} \ket{b} \\
	=& \sum_{nmlk} \bra{b} a_n \rangle \bra{b_n} b_m \rangle \bra{a_m} \hat{O} \ket{a_l} \bra{b_l} b_k \rangle \bra{a_k} a \rangle \\
	=& \sum_{nl} \bra{b} a_n \rangle \bra{a_n} \hat{O} \ket{a_l} \bra{a_l} b' \rangle
\end{align*}
\end{tcolorbox}

\textbf{习题1.14}

$\hat{A}$是厄米算符,幺正变换对其作用可写为
\[
	\hat{A}' = \hat{U} \hat{A} \hat{U}^\dagger
\]

证明$\hat{A}'$也为厄米算符,且它和$\hat{A}$有相同的本征值。
\begin{tcolorbox}[breakable, colback = black!5!white, colframe = black]
因为$\hat{A}$为厄米算符,则有
\[
	\hat{A}^\dagger = \hat{A}.
\]
则可以计算:
\[
	\hat{A}'^\dagger = (\hat{U}\hat{A}\hat{U}^\dagger)^\dagger = \hat{U} \hat{A}^\dagger \hat{U}^\dagger = \hat{U} \hat{A} \hat{U}^\dagger = \hat{A}'.
\]
对于$\hat{A}$的任意本征态$\ket{a}$,满足本征值方程
\[
	\hat{A} \ket{a} = a \ket{a}.
\]
都可以选取$\ket{a'} = \hat{U} \ket{a}$,可以计算:
\[
	\hat{A'} \ket{a'} = \hat{U}\hat{A}\hat{U}^\dagger \hat{U} \ket{a} = \hat{U} \hat{A} \ket{a} = a\hat{U}\ket{a} = a\ket{a'}.
\]
即$\hat{A'}$与$\hat{A}$具有相同本征值。
\end{tcolorbox}

\textbf{习题1.15}

证明$\hat{U} = \sum_n e^{i\theta_n}\ket{c_n}\bra{c_n} = e^{i\sum_n \theta_n\ket{c_n} \bra{c_n}} = e^{i\hat{A}}$。
\begin{tcolorbox}[breakable, colback = black!5!white, colframe = black]
令$\hat{A} = \sum_n \theta_n \ket{c_n} \bra{c_n}$,则有:
\[
	\hat{A}^2 = \sum_{n,n'} \theta_n \theta_n' \ket{c_n}\bra{c_n} c_n' \rangle \bra{c_n'} = \sum_n \theta_n^2 \ket{c_n}\bra{c_n}.
\]
即
\[
	\hat{A}^n = \sum_{m} \theta_m^n \ket{c_m} \bra{c_m}.
\]
则有:
\begin{align*}
	f(\hat{A}) =& e^{i\hat{A}} = 1 + \sum_{n=1}^\infty \frac{1}{n!} i^n \hat{A}^n \\
	=& \sum_{n=0}^\infty \frac{i^n}{n!} \sum_{m} \theta_m^n \ket{c_m} \bra{c_m} \\
	=& \sum_m \ket{c_m}\bra{c_m} \sum_{n=0}^\infty \frac{(i\theta_m)^n}{n!} \\
	=& \sum_m e^{i\theta_m} \ket{c_m} \bra{c_m} \\
	=& \sum_n e^{i\theta_n} \ket{c_n} \bra{c_n}.
\end{align*}
\end{tcolorbox}

\textbf{习题1.16}

计算$\hat{\sigma}_y$表象下Pauli矩阵的表示。
\begin{tcolorbox}[breakable, colback = black!5!white, colframe = black]
Pauli矩阵在$\hat{\sigma}_z$算符本征基矢下的表示分别为:
\[
	\sigma_x = \left(
	\begin{matrix}
		0 & 1 \\
		1 & 0
	\end{matrix}\right), ~~
	\sigma_y = \left(
	\begin{matrix}
		0 & -i \\
		i & 0
	\end{matrix}\right), ~~
	\sigma_z = \left(
	\begin{matrix}
		1 & 0 \\
		0 & -1
	\end{matrix}\right).
\]
将$\sigma_y$对角化,可以计算其本征值满足方程为:
\[
	\mathbf{det}(\sigma_y - \lambda \mathbf{I}) = \left\vert 
	\begin{matrix}
		-\lambda & -i \\
		i & -\lambda
	\end{matrix}\right\vert = \lambda^2 - 1 = 0.
\]
即两个本征值为$\lambda_1 = 1, \lambda_2 = -1$。

对应本征值$\lambda_1 = 1$的本征态矢量为:$u_1 = \left(\begin{matrix} -\frac{i}{\sqrt{2}} & \frac{1}{\sqrt{2}} \end{matrix}\right)^T$;

对应本征值$\lambda_2 = -1$的本征态矢量为:$u_2 = \left(\begin{matrix} \frac{i}{\sqrt{2}} & \frac{1}{\sqrt{2}} \end{matrix}\right)^T$。

即幺正矩阵$U = \left(\begin{matrix}
	-\frac{i}{\sqrt{2}} & \frac{i}{\sqrt{2}} \\
	\frac{1}{\sqrt{2}} & \frac{1}{\sqrt{2}}
\end{matrix}\right)$满足:
\[
	UU^\dagger = 1; ~~ U^\dagger\sigma_y U = \left(\begin{matrix}
		1 & 0 \\
		0 & -1
	\end{matrix}\right).
\]
同样的,可以计算:
\[
	U^\dagger \sigma_x U = \left( \begin{matrix}
		0 & i \\
		-i & 0
	\end{matrix}\right), ~~
	U^\dagger \sigma_z U = \left( \begin{matrix}
		0 & -1 \\
		-1 & 0
	\end{matrix}\right).
\]
综上所述,在$\hat{\sigma}_y$表象下,Pauli矩阵的表示为:
\[
	(\sigma_x)_y = \left( \begin{matrix}
		0 & i \\
		-i & 0
	\end{matrix}\right), ~~
	(\sigma_y)_y = \left( \begin{matrix}
		1 & 0 \\
		0 & -1
	\end{matrix}\right), ~~
	(\sigma_z)_y = \left( \begin{matrix}
		0 & -1 \\
		-1 & 0
	\end{matrix}\right).
\]
\end{tcolorbox}

\textbf{习题1.17}

试求在动量$\hat{p}$表象下算符$\hat{x}$和$\hat{p}$的表示。
\begin{tcolorbox}[breakable, colback = black!5!white, colframe = black]
在动量表象下,动量算符的表示为:
\[
	\bra{p} \hat{p} \ket{p'} = p\delta(p-p').
\]
在动量表象下,坐标算符的表示为:
\begin{align*}
	\bra{p} \hat{x} \ket{p'} =& \int dx x\bra{p} x \rangle \bra{x} p' \rangle \\
	=& \frac{1}{2\pi\hbar} \int dx x e^{ix(p'-p)/\hbar} \\
	=& \frac{i}{2\pi} \int dx \frac{\partial}{\partial p} e^{ix(p'-p)/\hbar} \\
	=& \frac{i}{2\pi} \frac{\partial}{\partial p} \int dx e^{ix(p'-p)/\hbar} \\
	=& i\hbar \frac{\partial}{\partial p} \delta(p'-p).
\end{align*}
\end{tcolorbox}

\textbf{习题1.18}

证明$\bra{n} m \rangle = \delta_{mn}$。
\begin{tcolorbox}[breakable, colback = black!5!white, colframe = black]
由$\ket{n} = \frac{\hat{a}^{\dagger n}}{\sqrt{n!}}\ket{0}$,可以计算:
\[
	\bra{n} m \rangle = \bra{0} \frac{\hat{a}^n \hat{a}^{\dagger m}}{\sqrt{n!m!}} \ket{0}.
\]
可以计算:
\begin{align*}
	\hat{a}\hat{a}^\dagger =& \hat{n}+1; \\
	\hat{a}\hat{a}\hat{a}^\dagger\hat{a}^\dagger =& \hat{a}(\hat{n}+1)\hat{a}^\dagger \\
	=& (\hat{n}+1)\hat{a}\hat{a}^\dagger + \hat{a}\hat{a}^\dagger \\
	=& (\hat{n}+2)\hat{a}\hat{a}^\dagger \\
	=& (\hat{n}+2)(\hat{n}+1)
\end{align*}
假设当$n=k$时,有:
\[
	\hat{a}^k\hat{a}^{\dagger k} = \prod_{i=1}^k(\hat{n}+i).
\]
可以验证,当$n=k+1$时,有:
\begin{align*}
	\hat{a}^{k+1}\hat{a}^{\dagger (k+1)} =& \hat{a}\hat{a}^k\hat{a}^{\dagger k}\hat{a}^\dagger \\
	=& \hat{a} \prod_{i=1}^k(\hat{n}+i) \hat{a}^\dagger \\
	=& \prod_{i=1}^k(\hat{n}+1+i)\hat{a}\hat{a}^\dagger \\
	=& \prod_{i=2}^{k+1}(\hat{n}+i)(\hat{n}+1) \\
	=& \prod_{i=1}^{k+1} (\hat{n}+i).
\end{align*}
则容易得到,当$n=m$时,有:
\[
	\langle n \vert n \rangle = \bra{0} \frac{\prod_{i=1}^n (\hat{n}+1)}{n!} \ket{0} = \frac{n!}{n!} \bra{0} 0 \rangle = 1.
\]
当$n\neq m$时,不妨取$n < m$,令$l=m-n>0$。则有:
\[
	\bra{n} m \rangle = \bra{0} \frac{\hat{a}^n\hat{a}^{\dagger (n+l)}}{\sqrt{n!m!}} \ket{0} = \frac{1}{\sqrt{n!m!}} \bra{0} \prod_{i=1}^n (\hat{n}+i) \hat{a}^{\dagger l} \ket{0} = \sqrt{\frac{n!}{m!}} \bra{0} \hat{a}^{\dagger l} \ket{0} = 0.
\]
其中利用了$\bra{0} \hat{a}^\dagger = 0$.

同样的,当$n>m$时,令$s = n-m$。则有:
\[
	\bra{n} m \rangle = \bra{0} \frac{\hat{a}^{s+m}\hat{a}^{\dagger m}}{\sqrt{n!m!}} \ket{0} = \frac{1}{\sqrt{n!m!}} \bra{0} \hat{a}^s \prod_{i=1}^m (\hat{n}+i) \ket{0} = \sqrt{\frac{m!}{n!}} \bra{0} \hat{a}^s \ket{0} = 0.
\]
其中利用了$\hat{a} \ket{0} = 0$.

综上所述,$\bra{n} m \rangle = \delta_{mn}$。
\end{tcolorbox}

\textbf{习题1.19}

求粒子数表象下$\hat{a}$和$\hat{a}^\dagger$的表示。
\begin{tcolorbox}[breakable, colback = black!5!white, colframe = black]
注意到对于$\hat{a}$和$\hat{a}^\dagger$有:
\[
	\hat{n}\hat{a} \ket{n} = (n-1) \hat{a}\ket{n}, ~ \hat{n}\hat{a}^\dagger \ket{n} = (n+1) \hat{a}^\dagger\ket{n}, ~ \hat{n} \ket{n} = n \ket{n}.
\]
则有:
\[
	\hat{a} \ket{n} = \lambda_1 \ket{n-1}; ~ \hat{a}^\dagger \ket{n} = \lambda_2 \ket{n+1}.
\]
利用$\hat{n} = \hat{a}^\dagger \hat{a}$,可以计算:
\[
	\hat{n}\ket{n} = \hat{a}^\dagger \hat{a} \ket{n} = \lambda_2\lambda_1 \ket{n}
\]
利用正交归一性,则有:
\[
	\bra{n}\hat{a}^\dagger\hat{a} \ket{n} = \bra{n} \hat{n} \ket{n} = n = \vert \lambda_1 \vert^2 \bra{n-1} n-1 \rangle = \vert \lambda_1 \vert^2
\]
\[
	\bra{n}\hat{a}\hat{a}^\dagger\ket{n} = \bra{n}(\hat{n}+1)\ket{n} = n+1 = \vert \lambda_2 \vert^2 \bra{n+1} n+1 \rangle = \vert \lambda_2 \vert^2
\]
选取相位因子为$0$,则有:
\[
	\lambda_1 = \sqrt{n}, ~ \lambda_2 = \sqrt{n+1}.
\]
即:
\[
	\hat{a} \ket{n} = \sqrt{n} \ket{n-1}; ~ \hat{a}^\dagger \ket{n} = \sqrt{n+1} \ket{n+1}.
\]
则在粒子数表象下,$\hat{a}$和$\hat{a}^\dagger$的表示分别可以写为:
\begin{align*}
	\bra{m} \hat{a} \ket{n} =& \sqrt{n} \bra{m} n-1 \rangle = \sqrt{n} \delta_{m,n-1}; \\
	\bra{m} \hat{a}^\dagger \ket{n} =& \sqrt{n+1} \bra{m} n+1 \rangle = \sqrt{n+1} \delta_{m,n+1}.
\end{align*}
写成矩阵形式,为:
\begin{align*}
	a =& \left( \begin{matrix}
		0 & \sqrt{1} & 0 & 0 & \cdots \\
		0 & 0 & \sqrt{2} & 0 & \cdots \\
		0 & 0 & 0 & \sqrt{3} & \cdots \\
		0 & 0 & 0 & 0 & \cdots \\
		\vdots & \vdots & \vdots & \vdots & \vdots & 
	\end{matrix} \right); \\
	a^\dagger =& \left( \begin{matrix}
		0 & 0 & 0 & 0 & \cdots \\
		\sqrt{1} & 0 & 0 & 0 & \cdots \\
		0 & \sqrt{2} & 0 & 0 & \cdots \\
		0 & 0 & \sqrt{3} & 0 & \cdots \\
		\vdots & \vdots & \vdots & \vdots & \vdots & 
	\end{matrix} \right).
\end{align*}
\end{tcolorbox}

\textbf{习题1.20}

求本征波函数$\langle x \vert n \rangle$。
\begin{tcolorbox}[breakable, colback = black!5!white, colframe = black]
在粒子数表象下,基矢$\ket{n}$可以由$\hat{a}^\dagger$不断作用在$\ket{0}$上得到,即:
\[
	\ket{n} = \frac{(\hat{a}^\dagger)^n}{\sqrt{n!}} \ket{0}.
\]
利用$\hat{a}\ket{0} = 0$,有:
\[
	\bra{x} \hat{a} \ket{0} = \frac{1}{\sqrt{2}}\left( \frac{x}{x_0}+\frac{\hbar}{p_0} \frac{\partial}{\partial x} \right) H_0(x) = 0.
\]
其中$\varphi_0(x) = \langle x \vert 0 \rangle$。可以解得:
\[
	\varphi_0(x) = Ce^{-\frac{M\omega}{2\hbar}x^2}.
\]
其中$C$为归一化系数。利用归一性,可以计算系数$C$:
\[
	\int_{-\infty}^\infty \vert \varphi_0(x) \vert^2 \,dx = \vert C \vert^2 \int_{-\infty}^\infty e^{-\frac{M\omega}{\hbar}x^2}\,dx = \vert C \vert^2 \sqrt{\frac{\pi\hbar}{M\omega}} = 1.
\]
取相位因子为$0$,则有:
\[
	\varphi_0(x) = \left( \frac{M\omega}{\pi\hbar} \right)^{1/4} e^{-\frac{M\omega}{2\hbar}x^2}.
\]
则本征波函数为:
\begin{align*}
	\bra{x} n \rangle =& \bra{x} \frac{(\hat{a}^\dagger)^n}{\sqrt{n!}} \ket{0} = \frac{1}{\sqrt{2^nn!}} \left( \frac{x}{x_0}-\frac{\hbar}{p_0} \frac{d}{d x} \right)^n \varphi_0(x) \\
	=& \sqrt{\frac{\hbar^n}{(2M\omega)^nn!}} \left( \frac{M\omega}{\hbar} x - \frac{d}{d x} \right)^n \varphi_0(x).
\end{align*}
可以计算:
\[
	\varphi_1(x) = \left[ \frac{4}{\pi}\left(\frac{M\omega}{\hbar}\right)^3 \right]^{1/4} xe^{-\frac{M\omega}{2\hbar} x^2},~ \varphi_2(x) = \left( \frac{M\omega}{4\pi\hbar} \right)^{1/4} \left( \frac{2M\omega}{\hbar}x^2 - 1 \right) e^{-\frac{M\omega}{2\hbar}x^2}.
\]
\end{tcolorbox}
\end{document}

\begin{tcolorbox}[breakable, colback = black!5!white, colframe = black]

\end{tcolorbox}