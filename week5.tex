\documentclass[reqno,a4paper,12pt]{amsart}

\usepackage{amsmath,amssymb,amsthm,geometry,xcolor,soul,graphicx}
\usepackage{titlesec}
\usepackage{enumerate}
\usepackage{lipsum}
\usepackage{listings}
\RequirePackage[most]{tcolorbox}
\usepackage{braket}
\allowdisplaybreaks[4] %align公式跨页
\usepackage{xeCJK}
\setCJKmainfont[AutoFakeBold = true]{Kai}
\geometry{left=0.7in, right=0.7in, top=1in, bottom=1in}

\renewcommand{\baselinestretch}{1.3}

\title{高等量子力学第五周作业}
\author{董建宇 ~~ 202328000807038}

\begin{document}

\maketitle
\titleformat{\section}[hang]{\small}{\thesection}{0.8em}{}{}
\titleformat{\subsection}[hang]{\small}{\thesubsection}{0.8em}{}{}

\textbf{习题1:}

取$L_1 = \sigma_z, \ \gamma_1 = \gamma, \ \gamma_{\alpha\geq 2} = 0, \ H = 0$, 求解Lindblad方程。
\[
	\dot{\rho}(t) = \gamma \left( \sigma_z\rho\sigma_z - \frac{1}{2}\{ \sigma_z\sigma_z, \rho \} \right)
\]

讨论其稳态性质。
\begin{tcolorbox}[breakable, colback = black!5!white, colframe = black]
二能级系统密度矩阵总可以写作如下形式:
\[
	\rho = \frac{\mathbf{I} + \vec{r} \cdot \vec{\sigma}}{2}.
\]
则对时间求导可得:
\[
	\dot{\rho} = \frac{1}{2} \vec{v}\cdot\vec{\sigma}.
\]
可以计算:
\[
	\sigma_z \rho = \frac{\sigma_z+ix\sigma_y-iy\sigma_x+z\mathbf{I}}{2}.
\]
\[
	\sigma_z\rho\sigma_z = \frac{\mathbf{I} - x\sigma_x - y\sigma_y + z\sigma_z}{2}.
\]
\[
	\{ \sigma_z\sigma_z, \rho \} = \{ \mathbf{I}, \rho \} = 2\rho.
\]
\[
	\sigma_z\rho\sigma_z - \frac{1}{2}\{\sigma_z\sigma_z, \rho\} = -(x\sigma_x+y\sigma_y).
\]
则Lindblad方程为:
\[
	\dot{\rho} = \frac{1}{2}(v_x\sigma_x+v_y\sigma_y+v_z\sigma_z) = -\gamma(x\sigma_x+y\sigma_y).
\]
其中$v_i = \dot{x}_i = \frac{dx_i}{dt}$,其中$i = x, y, z$。则有如下常微分方程:
\begin{align*}
	\frac{dx}{dt} =& -2\gamma x; \\
	\frac{dy}{dt} =& -2\gamma y; \\
	\frac{dz}{dt} =& 0.
\end{align*}
则有:
\[
	x(t) = x_0 e^{-2\gamma t}; \ \ y(t) = y_0 e^{-2\gamma t}; \ \ z(t) = z_0.
\]
当$t\to \infty$时,有:
\[
	\lim_{t\to\infty} x(t) = 0; \ \ \lim_{t\to\infty} y(t) = 0; \ \ \lim_{t\to\infty} z(t) = z_0.
\]
即当时间趋向于无穷时,体系状态处于Bloch球z轴上,且具体位置取决于体系初始状态。
\end{tcolorbox}

\textbf{习题2:}

证明算符$\hat{A} = a\hat{L}+b\hat{S}$满足角动量算符对易关系,当且仅当$a=b=1$或$a=1,b=0$或$a=0,b=1$。

\begin{tcolorbox}[breakable, colback = black!5!white, colframe = black]
可以计算:
\begin{align*}
	[\hat{A}_\alpha, \hat{A}_\beta] =& [a\hat{L}_\alpha+b\hat{S}_\alpha, a\hat{L}_\beta+b\hat{S}_\beta] \\
	=& a^2[\hat{L}_\alpha, \hat{L}_\beta] + b^2[\hat{S}_\alpha, \hat{S}_\beta] \\
	=& a^2 i\hbar \epsilon_{\alpha\beta\gamma} \hat{L}_\gamma + b^2 i\hbar \epsilon_{\alpha\beta\gamma} \hat{S}_\gamma \\
	=& i\hbar\epsilon_{\alpha\beta\gamma} (a^2\hat{L}_\gamma+b^2\hat{S}_\gamma) \\
	=& i\hbar\epsilon_{\alpha\beta\gamma} \hat{A}_\gamma \\
	=& i\hbar\epsilon_{\alpha\beta\gamma} (a\hat{L}_\gamma+b\hat{S}_\gamma).
\end{align*}
则有:
\[
	a^2 = a, \ \ b^2 = b.
\]
即$a$和$b$的取值为$0$或$1$。当$a=b=0$时,算符为$0$没有意义,因此当算符$\hat{A} = a\hat{L}+b\hat{S}$满足角动量对易关系,当且仅当$a=b=1$或$a=1,b=0$或$a=0,b=1$。
\end{tcolorbox}

\textbf{习题3:}

写出自旋$3/2$算符$S_\alpha, \ (\alpha = x, y, z)$的矩阵表示。

\begin{tcolorbox}[breakable, colback = black!5!white, colframe = black]
选取$\hat{S}^2$和$\hat{S}_z$的共同本征基矢,则有:
\begin{align*}
	\bra{l'm'} \hat{S}_z \ket{lm} =& m\hbar \delta_{ll'} \delta_{mm'}; \\
	\bra{l'm'} \hat{S}_x \ket{lm} =& \frac{1}{2} \bra{l'm'} (\hat{S}_+ + \hat{S}_-) \ket{lm} \\
	=& \frac{\hbar}{2} (\sqrt{(l-m)(l+m+1)} \delta_{ll'} \delta_{m',m+1} + \sqrt{(l+m)(l-m+1)} \delta_{ll'} \delta_{m',m-1}); \\
	\bra{l'm'} \hat{S}_y \ket{lm} =& \frac{1}{2i} \bra{l'm'} (\hat{S}_+ - \hat{S}_-) \ket{lm} \\
	=& \frac{\hbar}{2i} (\sqrt{(l-m)(l+m+1)} \delta_{ll'} \delta_{m',m+1} - \sqrt{(l+m)(l-m+1)} \delta_{ll'} \delta_{m',m-1}).
\end{align*}
则矩阵表示为:
\begin{align*}
	S_z =& \frac{\hbar}{2} \left( \begin{matrix}
		3 & 0 & 0 & 0 \\
		0 & 1 & 0 & 0 \\
		0 & 0 & -1 & 0 \\
		0 & 0 & 0 & -3
	\end{matrix} \right); \\ 
	S_x =& \frac{\hbar}{2} \left( \begin{matrix}
		0 & \sqrt{3} & 0 & 0 \\
		\sqrt{3} & 0 & 2 & 0 \\
		0 & 2 & 0 & \sqrt{3} \\
		0 & 0 & \sqrt{3} & 0
	\end{matrix} \right); \\ 
	S_y =& \frac{\hbar}{2} \left( \begin{matrix}
		0 & -\sqrt{3}i & 0 & 0 \\
		\sqrt{3}i & 0 & -2i & 0 \\
		0 & 2i & 0 & -\sqrt{3}i \\
		0 & 0 & \sqrt{3}i & 0
	\end{matrix} \right).
\end{align*}
\end{tcolorbox}
\end{document}

\begin{tcolorbox}[breakable, colback = black!5!white, colframe = black]

\end{tcolorbox}