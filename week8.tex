\documentclass[reqno,a4paper,12pt]{amsart}

\usepackage{amsmath,amssymb,amsthm,geometry,xcolor,soul,graphicx}
\usepackage{titlesec}
\usepackage{enumerate}
%\usepackage{lipsum} used a paragraph to test the environment 
\usepackage{listings}
%\RequirePackage[most]{tcolorbox}
\usepackage{braket}
\allowdisplaybreaks[4] %align公式跨页
\usepackage{xeCJK}
\setCJKmainfont[AutoFakeBold = true]{Kai}
\geometry{left=0.7in, right=0.7in, top=1in, bottom=1in}

\renewcommand{\baselinestretch}{1.3}

\title{高等量子力学第七次作业}
\author{董建宇 ~~ 202328000807038}

%\setlength{\parindent}{2pt}

\begin{document}

\maketitle

\textbf{习题4.4}

求最低级Born近似(n=0)下微分散射截面的表达式。

\begin{proof}

在最低级Born近似下,$n=0$,即有:
\[
	T \ket{k} = V \sum_{n=0}^\infty \left( \frac{1}{E_k-H_0+i\varepsilon} V \right)^n \ket{k} = V \ket{k}.
\]

则有:
\[
	f(k\vec{e}_r, k\vec{e}_z) = -(2\pi)^2 M \bra{k\vec{e}_r} T \ket{k\vec{e}_z} = -(2\pi)^2 M \bra{k\vec{e}_r} V \ket{k\vec{e}_z}.
\]

从而微分散射截面可以写为:
\[
	\frac{d\sigma}{d\Omega} = \vert f(k\vec{e}_r, k\vec{e}_z) \vert^2 = M^2(2\pi)^4 \vert \bra{k\vec{e}_r} V \ket{k\vec{e}_z} \vert^2.
\]

\end{proof}

\medskip

\textbf{习题4.5}

验证方程
\[
	\bra{E_p \vec{e}_p} Elm \rangle = \delta(E_p - E) \bra{\vec{e}_p} lm \rangle.
\]

等价于等式
\[
	e^{ikz} = \sum_l i^l(2l+1) j_l(kr) P_l(\cos\theta).
\]

\begin{proof}
注意到:
\[
	\ket{E_p \vec{e}_p} = \sqrt{Mp} \ket{\vec{p}}.
\]

可以计算:
\begin{align*}
	\bra{E_p \vec{e}_p} Elm \rangle =& \sqrt{Mp} \bra{\vec{p}} Elm \rangle \\
	=& \sqrt{Mp} \int d\vec{r} \bra{\vec{p}} \vec{r} \rangle \bra{\vec{r}} Elm \rangle \\
	=& \frac{\sqrt{Mp}}{(2\pi)^{3/2}} \int d\vec{r} e^{-i\vec{p} \cdot \vec{r}} \bra{\theta\phi} lm \rangle R_{kl}(r).
\end{align*}

注意到,可以计算:
\begin{align*}
	&\frac{\sqrt{Mp}}{(2\pi)^{3/2}} \int r^2dr d\Omega_r \left( \sum_{l'} (-i)^{l'} (2l'+1) j_{l'}(pr) P_{l'}(\cos\theta) \right) i^l \sqrt{\frac{2Mk}{\pi}} j_l(kr) \bra{\vec{e}_r} lm \rangle \\
	=& \frac{\sqrt{Mp}}{(2\pi)^{3/2}} \int r^2dr d\Omega_r \sum_{l'} (-i)^{l'} j_{l'}(pr) 4\pi \sum_{m'} \bra{\vec{e}_p} l'm' \rangle \bra{l'm'} \vec{e}_r \rangle i^l \sqrt{\frac{2Mk}{\pi}} j_l(kr) \bra{\vec{e}_r} lm \rangle.
\end{align*}

其中,利用完备性关系与正交性关系
\[
	\int d\Omega_r \ket{\vec{e}_r} \bra{\vec{e}_r} = 1; \ \ \ \bra{l'm'} lm \rangle = \delta_{ll'} \delta_{mm'}.
\]

上式可以简化为:
\begin{align*}
	&\frac{\sqrt{Mp}}{(2\pi)^{3/2}} \int r^2dr d\Omega_r \sum_{l'} (-i)^{l'} j_{l'}(pr) 4\pi \sum_{m'} \bra{\vec{e}_p} l'm' \rangle \bra{l'm'} \vec{e}_r \rangle i^l \sqrt{\frac{2Mk}{\pi}} j_l(kr) \bra{\vec{e}_r} lm \rangle \\
	=& \frac{\sqrt{Mp}}{(2\pi)^{3/2}} \sqrt{\frac{2Mk}{\pi}} 4\pi \int dr r^2 j_l(pr) k_l(kr) \bra{\vec{e}_p} lm \rangle \\
	=& \frac{2M\sqrt{pk}}{\pi} \frac{\pi}{2k^2}\delta(p-k) \bra{\vec{e}_p} lm \rangle \\
	=& \frac{M}{k} \delta(p-k) \bra{\vec{e}_p} lm \rangle \\
	=& \delta(E_p - E) \bra{\vec{e}_p} lm \rangle.
\end{align*}

其中$E_p = \frac{p^2}{2M}, E = \frac{k^2}{2M}$,取自然单位制$\hbar = 1$,则有:
\[
	\delta(E_p - E) = \delta(\frac{1}{2M}(p^2-k^2)) = \frac{M}{p} \delta(p-k).
\]

若有:
\[
	\bra{E_p \vec{e}_p} Elm \rangle = \delta(E_p-E) \bra{\vec{e}_p} lm \rangle.
\]

则不难发现,有:
\[
	e^{-i\vec{p}\cdot\vec{r}} = \sum_l (-i)^l (2l+1) j_l(pr) P_l(\cos\theta).
\]

两侧取共轭,得:
\[
	e^{i\vec{p} \cdot \vec{r}} = \sum_l i^l (2l+1) j_l(pr) P_l(\cos\theta).
\]

其中$\theta = \langle\vec{p}, \vec{r}\rangle$,即$\vec{p}$与$\vec{r}$的夹角。
\end{proof}

\medskip

\textbf{习题4.6}

在低能散射中通常只考虑较低的几个分波,如$s$波和$p$波等。试分析其中的道理,给出只需要考虑$s$波的条件。

\begin{proof}
对于自由球面波$\varphi_{k,l,m}^{(0)}(\vec{r})$,围绕$(\theta_0, \phi_0)$方向取无限小立体角$d\Omega_0$,在$r$到$r+dr$间隔内找到粒子的概率正比于
\[
	r^2 j_l^2(kr) \vert Y_l^m(\theta_0, \phi_0) \vert^2 drd\Omega_0.
\]

对于趋向零的$\rho$而言,有:
\[
	j_l(\rho) \sim \frac{\rho^l}{(2l+1)!!}.
\]

则上式中概率在原点附近的行为和$r^{2l+2}$的行为相同,因而$l$越大,概率增加越缓慢。$\rho^2 j_l^2(\rho)$只要
\[
	\rho < \sqrt{l(l+1)},
\]

函数就将保持在很小的数值。因此,如果
\[
	r < \frac{1}{k}\sqrt{l(l+1)}
\]

我们可以认为概率表达式实际上等于零,即处在$\ket{\varphi_{k,l,m}^{(0)}}$态的一个粒子对于以$O$为圆心,以
\[
	b_l(k) = \frac{1}{k}\sqrt{l(l+1)}
\]

为半径的球内发生的过程毫无反应。

对于任意中心势$V(r)$,分波$\varphi_{k,l,m}(\vec{r})$都具有如下形式:
\[
	\varphi_{k,l,m}(\vec{r}) = R_{k,l}(r) Y_l^m (\theta, \phi) = \frac{1}{r} u_{k,l}(r) Y_l^m(\theta, \phi).
\]

其中,$u_{k,l}$满足:
\[
	\left[ -\frac{\hbar^2}{2\mu}\frac{d^2}{dr^2} + \frac{l(l+1)\hbar^2}{2\mu r^2} + V(r) \right]u_{k,l}(r) = \frac{\hbar^2k^2}{2\mu} u_{k,l}(r).
\]

且在原点处满足:
\[
	u_{k,l}(0) = 0.
\]

当$r\to \infty$时,有:
\[
	u_{k,l} \sim C \sin\left( kr - l\frac{\pi}{2} + \delta_l \right).
\]

则分波$\phi_{k,l,m}(\vec{r})$可以写作:
\[
	\varphi_{k,l,m}(\vec{r}) \sim -CY_l^m(\theta, \phi) \frac{e^{-ikr} e^{i\left( l\frac{\pi}{2}-\delta_l \right)} - e^{ikr} e^{-i\left( l\frac{\pi}{2}-\delta_l \right)}}{2ir}
\]

即一个向内波和一个向外波的叠加。可以重新定义一个分波$\bar{\varphi}_{k,l,m}(\vec{r}) = \varphi_{k,l,m}(\vec{r}) e^{i\delta_l}$,并选择常数$C$,使得:
\[
	\bar{\varphi}_{k,l,m}(\vec{r}) \sim -Y_l^m(\theta, \varphi) \frac{e^{-ikr} e^{il\pi/2} - e^{ikr} e^{-il\pi/2} e^{2i\delta_l}}{2ikr}.
\]

即相移$2\delta_l$的相位因子$e^{2i\delta_l}$纳入了势场对角动量量子数为$l$的粒子的全部影响。则考虑有限射程为$r_0$的势场$V(r)$,即
\[
	V(r) = 0, \ \ \ r>r_0.
\]

则类似自由球面波,对于满足$b_l(k) >> r$的那些波,实际上感受不到势场的作用,因为对应的内向波在到达势场作用范围前就已经反向传播了。于是,对能量的每一个值都存在着一个角动量临界量子数$l_M$,近似满足:
\[
	\sqrt{l_M(l_M + 1)} \sim kr_0.
\]

当$l\leq l_M$时,相移$\delta_l$才有显著的影响。即当入射能量较低,势场射程较短时,$l_M$较小,即此时只需考虑较低的几个分波,如$s$波和$p$波等。当只需考虑$s$波时,有:
\[
	0 \leq l_M < 1.
\]

即近似的有:
\[
	kr_0 < \sqrt{2}.
\]
\end{proof}


\end{document}