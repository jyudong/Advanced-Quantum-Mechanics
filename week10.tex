\documentclass[reqno,a4paper,12pt]{amsart}

\usepackage{amsmath,amssymb,amsthm,geometry,xcolor,soul,graphicx}
\usepackage{titlesec}
\usepackage{enumerate}
%\usepackage{lipsum} used a paragraph to test the environment 
\usepackage{listings}
%\RequirePackage[most]{tcolorbox}
\usepackage{braket}
\allowdisplaybreaks[4] %align公式跨页
\usepackage{xeCJK}
\setCJKmainfont[AutoFakeBold = true]{Kai}
\geometry{left=0.7in, right=0.7in, top=1in, bottom=1in}

\renewcommand{\baselinestretch}{1.3}

\title{高等量子力学第十次作业}
\author{董建宇 ~~ 202328000807038}

%\setlength{\parindent}{2pt}

\begin{document}

\maketitle

\textbf{习题5.3}

证明方程
\[
	\sum_{i, j} \ket{\phi_\alpha(i), \phi_\delta(j)} \bra{\phi_\beta(i), \phi_\tau(j)} = \hat{a}^\dagger_\alpha \hat{a}^\dagger_\delta \hat{a}_\tau \hat{a}_\beta.
\]

\begin{proof}
在$\alpha \neq \delta \neq \beta \neq \tau$的情况下,不妨取$\alpha<\delta<\beta<\tau$,可以计算:
\begin{align*}
	&\sum_{i, j} \ket{\phi_\alpha(i), \phi_\delta(j)} \bra{\phi_\beta(i), \phi_\tau(j)} \ket{n_1, n_2, \cdots, n_M} \\
	=& \frac{1}{\sqrt{N!n_1!n_2!\cdots n_M!}} \sum_\sigma \sigma \sum_{i, j} \ket{\phi_\alpha(i), \phi_\delta(j)} \bra{\phi_\beta(i), \phi_\tau(j)} \psi^*(n_1, \cdots, n_M) \rangle \\
	=& \frac{1}{\sqrt{N!n_1!n_2!\cdots n_M!}} \sum_\sigma \sigma \ket{\psi^*(1, \cdots, n_\alpha+1, \cdots, n_\delta+1, \cdots, n_\beta-1, \cdots, n_\tau-1, \cdots, n_M)} \\
	=& \sqrt{(n_\alpha+1)(n_\delta+1)n_\beta n_\tau} \ket{1, \cdots, n_\alpha+1, \cdots, n_\delta+1, \cdots, n_\beta-1, \cdots, n_\tau-1, \cdots, n_M}.
\end{align*}

即有:
\[
	\hat{a}^\dagger_\alpha \hat{a}^\dagger_\delta \hat{a}_\tau \hat{a}_\beta = \sum_{i, j} \ket{\phi_\alpha(i), \phi_\delta(j)} \bra{\phi_\beta(i), \phi_\tau(j)}.
\]

对于其他$\alpha, \beta, \delta, \tau$的大小关系,可以得到同样的表达式。
\end{proof}

\medskip

\textbf{习题5.4}

对全同费米子系统,用产生湮灭算符的对易关系证明:
\begin{enumerate}
	\item $\alpha \neq \beta$, $[\hat{n}_\alpha, \hat{n}_\beta] = 0$.
	
	\item $\hat{n}_\alpha$的本征值只能取$0$或者$1$。
\end{enumerate}

\begin{proof}
\begin{enumerate}
\item 费米子产生湮灭算符满足反对易关系:
\[
	\{\hat{a}_i, \hat{a}^\dagger_j\} = \delta_{ij}; \ \ \{\hat{a}_i, \hat{a}_j\} = 0; \ \ \{\hat{a}^\dagger_i, \hat{a}^\dagger_j\} = 0.
\]

则当$i\neq j$时,有:
\[
	\hat{a}_i\hat{a}_j^\dagger = -\hat{a}_j^\dagger\hat{a}_i; \ \ \hat{a}_i \hat{a}_j = -\hat{a}_j\hat{a}_i; \ \ \hat{a}_i^\dagger\hat{a}_j^\dagger = - \hat{a}_j^\dagger \hat{a}_i^\dagger.
\]

则可以计算,当$\alpha\neq \beta$时:
\[
	[\hat{n}_\alpha, \hat{n}_\beta] = \hat{a}^\dagger_\alpha\hat{a}_\alpha \hat{a}_\beta^\dagger\hat{a}_\beta - \hat{a}_\beta^\dagger\hat{a}_\beta \hat{a}^\dagger_\alpha\hat{a}_\alpha = (-1)^4 \hat{a}_\beta^\dagger\hat{a}_\beta \hat{a}^\dagger_\alpha\hat{a}_\alpha - \hat{a}_\beta^\dagger\hat{a}_\beta \hat{a}^\dagger_\alpha\hat{a}_\alpha = 0.
\]

\item 注意到,对于费米子系统,有:
\[
	\{ \hat{a}_\alpha^\dagger, \hat{a}_\beta^\dagger\} = 0.
\]

则当$\alpha=\beta$时,$2\hat{a}_\alpha^\dagger\hat{a}_\alpha^\dagger = 0$,从而有:
\[
	\hat{a}_\alpha^\dagger \hat{a}_\alpha^\dagger \ket{0} = 0.
\]

对于量子态$\ket{n_1, n_2, \cdots, n_M}$,可以由产生算符作用在真空态上得到,即
\[
	\ket{n_1, n_2, \cdots, n_M} \equiv {a_1^\dagger}^{n_1}{a_2^\dagger}^{n_2}\cdots{a_M^\dagger}^{n_M} \ket{0}.
\]

但当$n_i\geq2$时,$\ket{n_1, n_2, \cdots, n_M} = 0$,当$n_i=0$或$1$时,$\hat{n}_i$的本征值为$0$或$1$。综上所述,$\hat{n}_\alpha$的本征值只能取$0$或者$1$。
\end{enumerate}
\end{proof}

\medskip

\textbf{习题5.5}

求黑体辐射中光子的化学势。

\begin{proof}
由于黑体辐射中光子数不守恒,即产生或湮灭一个光子体系能量不发生变化,则黑体辐射中光子的化学势为$0$。

对于黑体辐射,可以计算巨配分函数为:
\[
	\ln \varXi = \frac{\pi^2V}{45c^3\hbar^3} k_B^3 T^3.
\]

内能为:
\[
	U = -\frac{\partial \ln \varXi}{\partial \beta} = \frac{\pi^2V}{15c^3\hbar^3} k_B^4T^4.
\]

压强为:
\[
	p = \frac{1}{3}\frac{U}{V} = \frac{\pi^2}{45c^3\hbar^3} k_B^4T^4.
\]

自由能为:
\[
	F = -\frac{1}{3}U.
\]

则Gibbs自由能为:
\[
	G = F+pV = 0 = N\mu.
\]

则化学势为0.
\end{proof}

\medskip

\textbf{习题5.6}

当光场的单个模式处在相干态$\ket{\alpha_{k\vec{e}_z, \vec{e}_x}}$态时,求对应电场和磁场算符在该态上的平均值。

\begin{proof}
电磁场算符分别为:
\begin{align*}
	&\vec{E} = \sum_{\vec{k}, a} \sqrt{\frac{\hbar\omega_k}{2\epsilon_0 V}} \vec{e}_{\vec{k}, a} i (\hat{a}_{\vec{k}, a} e^{i\vec{k}\cdot\vec{r}} - \hat{a}_{\vec{k}, a}^\dagger e^{-i\vec{k}\cdot\vec{r}}) \\
	&\vec{B} = \sum_{\vec{k}, a} \sqrt{\frac{\hbar}{2\epsilon_0 V \omega_k}} (i\vec{k})\times \vec{e}_{\vec{k}, a} (\hat{a}_{\vec{k}, a} e^{i\vec{k}\cdot\vec{r}} - \hat{a}_{\vec{k}, a}^\dagger e^{-i\vec{k}\cdot\vec{r}}) 
\end{align*}

对于相干态$\ket{\alpha_{k\vec{e}_z, \vec{e}_x}}$,满足本征值方程:
\[
	\hat{a}_{\vec{k}, a} \ket{\alpha_{k\vec{e}_z, \vec{e}_x}} = \alpha_{k\vec{e}_z, \vec{e}_x} \ket{\alpha_{k\vec{e}_z, \vec{e}_x}} \delta_{a,\vec{e}_x} \delta_{\vec{k}, k\vec{e}_z}.
\]

则可以计算电场和磁场算符在该态上的平均值如下:
\begin{align*}
	&\bra{\alpha_{k\vec{e}_z, \vec{e}_x}} \hat{\vec{E}} \ket{\alpha_{k\vec{e}_z, \vec{e}_x}} \\
	=& \sum_{\vec{k}, a} \sqrt{\frac{\hbar\omega_k}{2\epsilon_0 V}} \vec{e}_{\vec{k}, a} i ( \bra{\alpha_{k\vec{e}_z, \vec{e}_x}} \hat{a}_{\vec{k}, a} \ket{\alpha_{k\vec{e}_z, \vec{e}_x}} e^{i\vec{k} \cdot \vec{r}} - \bra{\alpha_{k\vec{e}_z, \vec{e}_x}} \hat{a}^\dagger_{\vec{k}, a} \ket{\alpha_{k\vec{e}_z, \vec{e}_x}} e^{-i\vec{k} \cdot \vec{r}}) \\
	=& \sqrt{\frac{\hbar\omega_k}{2\epsilon_0 V}} \vec{e}_x i (\alpha_{k\vec{e}_z, \vec{e}_x} e^{ikz} - \alpha_{k\vec{e}_z, \vec{e}_x}^* e^{-ikz}); \\
	&\bra{\alpha_{k\vec{e}_z, \vec{e}_x}} \hat{\vec{B}} \ket{\alpha_{k\vec{e}_z, \vec{e}_x}} \\
	=& \sum_{\vec{k}, a} \sqrt{\frac{\hbar}{2\epsilon_0 V \omega_k}} (i\vec{k})\times\vec{e}_{\vec{k}, a} (\bra{\alpha_{k\vec{e}_z, \vec{e}_x}} \hat{a}_{\vec{k}, a} \ket{\alpha_{k\vec{e}_z, \vec{e}_x}} e^{i\vec{k}\cdot \vec{r}} - \bra{\alpha_{k\vec{e}_z, \vec{e}_x}} \hat{a}^\dagger_{\vec{k}, a} \ket{\alpha_{k\vec{e}_z, \vec{e}_x}} e^{-i\vec{k} \cdot \vec{r}}) \\
	=& \sqrt{\frac{\hbar}{2\epsilon_0 V \omega_k}} ik\vec{e}_y (\alpha_{k\vec{e}_z, \vec{e}_x} e^{ikz} - \alpha^*_{k\vec{e}_z, \vec{e}_x} e^{-ikz}).
\end{align*}

\end{proof}

\end{document}